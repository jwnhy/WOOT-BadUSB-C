\documentclass[conference]{IEEEtran}
\IEEEoverridecommandlockouts
% The preceding line is only needed to identify funding in the first footnote. If that is unneeded, please comment it out.

\makeatletter
\def\lst@makecaption{%
  \def\@captype{table}%
  \@makecaption
}
\makeatother

\newcommand{\tool}{\mbox{\textsc{BadUSB-C}}\xspace}

\newcommand{\outline}[1]{}
\newcommand{\mytodocyan}[1]{}


\newcommand{\hongyi}[1]{}
\newcommand{\mytodopink}[1]{}

\newcommand{\shuqing}[1]{}
\newcommand{\mytodopurple}[1]{}

\newcommand{\yechang}[1]{}
\newcommand{\mytodored}[1]{}

\newcommand{\linyou}[1]{}
\newcommand{\mytodoblue}[1]{}

\newcommand{\chaozu}[1]{}
\newcommand{\mytodogrey}[1]{}
\usepackage{enumerate}
\usepackage[shortlabels]{enumitem}

\newcommand{\fengwei}[1]{}

\newcommand{\mycircled}[1]{%
	\begin{tikzpicture}[baseline={(char.base)}]
		\node (char) {\small{#1}};
		\node[draw,circle,minimum size=10,inner sep=0pt,overlay] (char.center){};
	\end{tikzpicture}
}
\usepackage{textcomp}

\usepackage{tikz,pgf}
\usetikzlibrary{calc}
\makeatletter
\def\mfontsize{\f@size}
\newcommand{\circled[2]}{
	\tikzset{mystyle/.style={circle,#1,minimum size=10,inner sep=0pt}}
	\tikz[baseline=-3pt]
	{
		\node[mystyle] (char.center) {\vphantom{WAH1g}#2};
	}
}
\makeatother

\definecolor{myyellow}{RGB}{237,125,49}
\definecolor{myblue}{RGB}{91,155,213}

\usepackage[breaklinks=true,hidelinks,colorlinks=true,citecolor=blue,urlcolor=black,linkcolor=purple]{hyperref}
\usepackage{balance}

\usepackage{xspace}
\usepackage{cite}
\usepackage{amsmath,amssymb,amsfonts}
\usepackage{graphicx}
\usepackage{textcomp}
\usepackage{xcolor}
\usepackage{hyperref}
\usepackage{threeparttable}
\usepackage{colortbl}
\usepackage{graphicx}
\usepackage{listings}
\usepackage{color}
\usepackage{array}
\usepackage{float}
\usepackage{graphicx}
\usepackage{subfigure}
\usepackage{multirow}
\usepackage{colortbl}
\usepackage[linesnumbered,boxed]{algorithm2e}
\usepackage{algpseudocode}
\usepackage{framed}
\setlength\FrameSep{0.5em}
\usepackage{pifont}
\usepackage{enumitem}
\usepackage{balance}
\usepackage{url}
\usepackage[T1]{fontenc}
\usepackage[utf8]{inputenc}
\usepackage[font=small,labelfont=bf,tableposition=top]{caption}
\usepackage{booktabs}
\usepackage{tabularx}
\usepackage{diagbox}
\usepackage{algpseudocode}
\usepackage{amsmath}
\usepackage{listings}
\usepackage{makecell}
\usepackage{verbatim}
\usepackage{acronym}
\usepackage[super]{nth}

\lstset{
	frame=single,
	breaklines,
	language=python,
	basicstyle=\small,
}


\def\BibTeX{{\rm B\kern-.05em{\sc i\kern-.025em b}\kern-.08em
    T\kern-.1667em\lower.7ex\hbox{E}\kern-.125emX}}


\begin{document}

\title{\tool: Revisiting BadUSB with Type-C}

\author{Anonymous Authors}
%\author{\IEEEauthorblockN{1\textsuperscript{st} Given Name Surname}
%\IEEEauthorblockA{\textit{dept. name of organization (of Aff.)} \\
%\textit{name of organization (of Aff.)}\\
%City, Country \\
%email address or ORCID}
%\and
%\IEEEauthorblockN{2\textsuperscript{nd} Given Name Surname}
%\IEEEauthorblockA{\textit{dept. name of organization (of Aff.)} \\
%\textit{name of organization (of Aff.)}\\
%City, Country \\
%email address or ORCID}
%\and
%\IEEEauthorblockN{3\textsuperscript{rd} Given Name Surname}
%\IEEEauthorblockA{\textit{dept. name of organization (of Aff.)} \\
%\textit{name of organization (of Aff.)}\\
%City, Country \\
%email address or ORCID}
%\and
%\IEEEauthorblockN{4\textsuperscript{th} Given Name Surname}
%\IEEEauthorblockA{\textit{dept. name of organization (of Aff.)} \\
%\textit{name of organization (of Aff.)}\\
%City, Country \\
%email address or ORCID}
%\and
%\IEEEauthorblockN{5\textsuperscript{th} Given Name Surname}
%\IEEEauthorblockA{\textit{dept. name of organization (of Aff.)} \\
%\textit{name of organization (of Aff.)}\\
%City, Country \\
%email address or ORCID}
%\and
%\IEEEauthorblockN{6\textsuperscript{th} Given Name Surname}
%\IEEEauthorblockA{\textit{dept. name of organization (of Aff.)} \\
%\textit{name of organization (of Aff.)}\\
%City, Country \\
%email address or ORCID}
%}

\maketitle
\thispagestyle{plain}
\pagestyle{plain}


\newacro{USB}{Universal Serial Bus}
\newacro{HID}{Human Interface Device}
\newacro{UI}{User Interface}
\newacro{PnP}{Plug-and-Play}
\newacro{OEM}{Original Equipment Manufacturer}
\newacro{DoS}{Denial of Service}
\newacro{MHL}{Mobile High-Definition Link}
\newacro{URB}{USB Request Block}
\newacro{OCR}{Optical Character Recognition}
\newacro{GUI}{Graphical User Interface}

\begin{abstract}

The security of the \ac{USB} protocol has been paid extensive attention to because of its
    wide usage.  Due to the \textit{trust-by-default} characteristics, \ac{USB}
    security has caused severe problems.  For example, a well-known firmware
    attack, BadUSB, performs malicious operations on the victim hosts through
    disguising ordinary \ac{USB} devices as human interface devices like keyboards and
    mice.  However, BadUSB suffers from several limitations.  Attackers cannot
    obtain the status of \ac{UI} to conduct precise attacks and get the visual feedback of their attacks.  In this work, we
    extended BadUSB to support the new \ac{USB} Type-C features and proposed a
    multi-mode attack model, \tool.  It obtains UI status to make attacks more 
    precise and effective.  To the best of our knowledge, \tool is the first attack model
    utilizing \ac{USB} Type-C.  To validate the usability and effectiveness, we
    conducted extensive experiments and a user study on a real-life attack
    scenario.  10 volunteers are involved in, and 4,172 of 94,058 records
    obtained by \tool in the user study are related to sensitive information.
%\fengwei{What results?} \fengwei{what countermeasures do we propose? including
    %isolated UI rendering.}
We also discussed the recommended countermeasures for our attack model,
    including isolated UI rendering, which may be inspiring for future research
    on defense methods.
\end{abstract}

\begin{IEEEkeywords}
USB; BadUSB; Type-C; Attack
\end{IEEEkeywords}


\acresetall

\section{Introduction}
\label{sec:introduction}

The Universal Serial Bus (USB) protocol has become popular around the world
since its appearance, as it provides a unified and easy-to-use approach for a
large range of devices to communicate with each other.  From version 1.0 till
now, USB specification has evolved rapidly and offered more and more
functionalities.  Nowadays, devices with USB support are ubiquitous.

On the other side of the coin, the security of USB has caused serious problems.
The designers of the USB protocol did not care much about security issues as
they wanted to make an easy-to-use protocol.  There are more than four hundred
vulnerabilities referencing USB on CVE list~\cite{website:CVE-list}.  As a
result, many attackers exploit these vulnerabilities and the
\textit{trust-by-default} characteristics of USB to conduct attacks, which puts
the privacy and financial security of USB users in danger~\cite{sok}.

BadUSB is a well-known class of firmware attacks~\cite{badusb}.  These attacks
are conducted through modification of the device firmware, which will disguise
normal USB devices as other types of devices that are \textit{trust-by-default}
by the hosts.  Typical simulated devices include HID (short for human interface
devices, including keyboards, mice, etc.) and disks.  Utilizing BadUSB,
attackers can pretend themselves as normal users, typing malicious commands to
victim computers, downloading and executing malicious scripts, and copying out
private data from disks.  Such attacks can easily escape from traditional
anti-virus software since it is hard to distinguish them from normal USB
devices.

Despite the advantageous features of BadUSB, there exist several limitations as
follows.  (1) Attackers cannot conduct attacks precisely, which decreases the
capabilities of BadUSB attacks.  When performing attacks on another host, the
attackers can not obtain the current user interface (UI) status which limits
them from taking subsequent moves.  For example, it is hard for attackers to
locate specific functional UI patterns like buttons and links on victim
computers by disguising USB devices like mice.  That explains why typical
BadUSB attacks often only stay in the command line, using commands to download
malicious scripts for execution.  However, these attacks may be intercepted by
anti-virus software or firewall due to the usage of the host network.  (2) To
our best knowledge, existing BadUSB attacks only utilize the features of USB
2.0.  The release of USB 3.0 makes USB more powerful, with a higher
transmission rate for data and the support towards a larger range of
peripherals including DisplayPort, HDMI, PowerDelivery, etc.  BadUSB attacks
can become more effective with the help of newly supported features in USB 3.0.
(3) There have emerged multiple efficacious countermeasures after the
appearance of BadUSB.  For example, GoodUSB offers a defense method by limiting
the functions of USB devices to users' expectations~\cite{tian2015defending}.
It provides a graphical interface for users to describe the functionalities or
roles of the USB device and reject any usage beyond the description.
%\shuqing{Can't bypass USBCheckIn.}

In this work, we proposed our approach addressing the limitations mentioned
above and implemented a multi-mode attack model of USB, named \tool.  \tool
extends BadUSB to support the features of USB Type-C.  Since USB Type-C is
capable to transfer video stream data, \tool could obtain the information of
victim's graphical interface during attacks.  Combining it with the emulation
of traditional HIDs like keyboards and mice, attackers are capable of
performing precise attacks.  We did experiments to verify that \tool could also
bypass many countermeasures for BadUSB since they often rely on interaction
with the graphical interface.  Moreover, we implemented multiple attacking
modes of USB attacks based on our approach to verify its effectiveness,
including HID Emulator Mode, Video Capture Mode, and Full Control Mode.  To improve the
efficiency and performance of \tool, we designed a filtering algorithm to
preprocess the video data before network transmission.  Moreover, we conducted
a series of experiments for each attack mode as well as for different types of
devices, including smartphones, personal computers, and tablet computers, to
validate the usability of \tool.  We also conducted a user study for attacks in
sharing power banks, one of the application scenarios of \tool.  During the
user study, we obtained a large amount of sensitive data through \tool, since
users are usually unconscious of the necessity to check the USB devices they
plugged in for security purposes.  
10 volunteers are involved in the user study.
At last, we found that 4,796 of 94,058 records obtained by \tool are related to privacy information of users.
After the validation of our attack model, we
proposed several defense methods as countermeasures, including external
hardware authorization, distrust-by-default, etc.  It is worth noting that we
designed a method, called isolated UI rendering, to separate the user interface
into sensitive and insensitive layers.  Only content on the insensitive layer
will be passed to the insecure driver and thus rendered on the external
display, which protects content on the sensitive layer.
%\shuqing{Case study, evaluation, notable results.}

We summarize our key contributions as follows:

\begin{itemize} 
    
    \item To our best knowledge, this is the first work to utilize new features
	of USB Type-C.  The combination of new support with conventional BadUSB
	makes attacks more precise and effective.
	
    \item Our approach can bypass many previous countermeasures of BadUSB.
	
    \item We conducted multiple experiments and a user study to validate the
	usability and effectiveness of \tool.  We also proposed several
	countermeasures for our attack model, which are reasonable and
	insightful. 
	%\fengwei{Add countermeasures, in particular the isolation design.}
	%\shuqing{Isolated UI rendering is added in the last paragraph.}
\end{itemize}

The rest of this paper is structured as follows.  Section~\ref{sec:background}
provides the background of USB specification.  Section~\ref{sec:related_work}
introduces the existing work of USB security from the aspects of attacking and
defense respectively.  In Section~\ref{sec:badusb}, we present the threat model
and the overall implementation of \tool in three different modes.  The
experiments and user study we conducted are featured in
Section~\ref{sec:experiment}.  We present the possible countermeasures of \tool
in Section~\ref{sec:countermeasures}.  The limits and impacts of our approach
are discussed in Section~\ref{sec:discussion}, and the conclusion lies in
Section~\ref{sec:conclusion}.
%\shuqing{Fill in after the structure is finalized.}















\section{Background}
\label{sec:background}
%\shuqing{Maybe we could introduce HID devices somewhere.}
%\noindent\outline{USB Standard}\\

We first introduce the development of \ac{USB} specification and emphasize the key
points adopted in this work. We also organize a brief timeline for introducing
key points of each protocol in Table~\ref{table:usb_timeline}.

%\noindent\outline{USB1.x}\\
Proposed in 1996, \ac{USB} 1.0~\cite{usb10} was developed to provide a unified
interface and thus reducing the cost of reconfiguring the software. It is
worth mentioning that as a polled-bus interface, all data transfers are
initiated by the host.

%\noindent\outline{HID Protocol}\\
Right after one year of the appearance of \ac{USB} 1.0, a standard named \acf{HID}~\cite{hid} was designed based on \ac{USB}. \ac{HID} is
designed to unify the implementation for devices like keyboards,
mice, etc. Before its appearance, the standard is divided among
manufacturers, for example, the mouse of Company A may use X-Y coordinates to
represent its location while the mouse of Company B uses relative displacement.
This means every device needs its own driver to work. After \ac{HID}, users only need to
write one driver for an entire class of \acp{HID}. Furthermore, the \ac{HID} standard also
requires all devices to be \ac{PnP}, which is indeed
convenient but insecure too.

In 1998, the first widely supported \ac{USB} protocol was designed. \ac{USB} 1.1~\cite{usb11}
provides two data transfer rates which are low speed (1.5 Mbit/s) and full
speed (12 MBit/s). At this point, due to the transfer limitation, it only supports
limited types of devices like keyboards, mice, etc.

%\noindent\outline{USB2.0}\\
In 2000, the \ac{USB} 2.0~\cite{usb20} specification was released. With high speed \mbox{(480
Mbit/s)} mode introduced, printers, cameras, CD-ROM drives, and network cards are
supported in this revision. Such a high data transfer rate also give rise to the
popularity of ``flash drive'', a portable device that allows physically
transferring data around~\cite{sok}. Although various peripherals are supported
in \ac{USB} 2.0, there is no reliable way to identify the type of device. This
security flaw allows attacks like BadUSB~\cite{badusb,rubber}.

%\noindent\outline{USB3.x}\\
\ac{USB} 3.0~\cite{usb30} was introduced in 2008, with a super speed \mbox{(5 Gbit/s)} data
transfer rate. Like its predecessor, more classes of peripherals are supported
in this revision. In 2013, the \ac{USB} Type-C connector standard was introduced as a
part of \ac{USB} 3.1~\cite{usb31}, providing a unified connector type for
PowerDelivery, Thunderbolt, DisplayPort, and HDMI.  Yet no improvement of
security is introduced in 3.x revisions, meaning any device claiming
itself as a monitor can capture the video stream from the host. Exposing
such a multi-purpose connector unprotected is insecure and allows attacks similar to \tool. In 2017, \ac{USB} 3.2~\cite{usb32} was released, doubling the data
transfer rate (20 Gbit/s).

%\noindent\outline{Connector Standard}
\begin{figure}[t]
    \centering
	\includegraphics[width=0.7\linewidth]{./Figs/usb_conn.png}
	\caption{\ac{USB} 1.x \& 2.x Connector.}
	\label{fig:usb_conn}
\end{figure}

As illustrated in Figure~\ref{fig:usb_conn}, the original \ac{USB} 1.x \& 2.x
connector only has two pins for data transferring \mbox{(D+ \& D-)}, which has
significantly limited data transfer rate \mbox{(5 Gbits/s Max)} and cannot support
peripherals like DisplayPort \mbox{(10.8 Gbit/s Min)}. Apart from that, support for
other peripherals also requires dedicated transferring lanes as their standards
are not compatible with \ac{USB} in most cases.  

\begin{figure}[t] 
	\centering
	\includegraphics[width=\linewidth]{./Figs/usb_c_conn.png} 
	\caption{\ac{USB} Type-C Connector.} 
	\label{fig:usb_c_conn} 
\end{figure}

Thus, to provide support towards a wider range of peripherals, a 24-pins
standard called \ac{USB} Type-C~\cite{typec} was introduced in 2013 by \ac{USB}-IF~\cite{usbif}. As it is designed to be double-sided, the number of actually usable
pins is halved. Nevertheless, this standard has largely enhanced the capability
of the \ac{USB} 3.x protocol. As presented in Figure~\ref{fig:usb_c_conn}, Type-C add
two high-speed data lanes \mbox{(TRX1 \& TRX2)} and keep the original data lane \mbox{(D+ \&
D-)}. The added lanes are used exclusively to support peripherals like
DisplayPort while the kept data lane transfers \ac{USB} packets.

%\noindent\outline{Security Problem}\\
During the development of the \ac{USB} specification, security was insufficiently considered~\cite{sok}. 
The \ac{USB}-IF believes it is the duty of \acp{OEM}
to decide whether security features should be implemented~\cite{usbsec}. But the divergent implementations give a chance for attacks like
BadUSB~\cite{rubber} and our \tool.

%\fengwei{FIXME: add a reference for each attack in the Table.}
%\fengwei{FIXME: The table is too wide, exceeds the max size.}
\begin{table*}
\begin{tabular}{|c|l|c|c|c|}
	\hline
	\textbf{Year} & \textbf{Protocol Version} & \textbf{Supported Peripherals} & \textbf{Transfer Speed} & \textbf{Attacks} \\
	\hline
	1996 & \ac{USB} 1.x~\cite{usb10,usb11} & Keyboard, Mouse... & 1.5 Mbit/s or 12 Mbit/s & \ac{HID} Emulation (BadUSB)~\cite{badusb} \\
	\hline
	2000 & \ac{USB} 2.0~\cite{usb20} & Flash Drive, High-Definition Link, CD Driver... & 480 Mbit/s & Autorun Attack~\cite{duqu}, Juice Filming~\cite{JFC,JFCImpact} \\
	\hline
	2008 & \ac{USB} 3.0~\cite{usb30} & / & 5 Gbit/s & / \\
	\hline
	2013 & \ac{USB} 3.1~\cite{usb31} & HDMI, DisplayPort, ThunderBolt... & 10 Gbit/s & \tool \\
	\hline
	2017 & \ac{USB} 3.2~\cite{usb32} & / & 20 Gbit/s & / \\
	\hline
\end{tabular}
	\linebreak
\caption{\ac{USB} Protocol Timeline.}
\label{table:usb_timeline}
\end{table*}

\section{Related Work}
\label{sec:related_work}
\hongyi{Move the stuff about USB protocol to Background}\\
\outline{Attack based on USB 1.0, briefly}\\
\outline{Attack based on USB 2.0, focus on works about application and transport layer}\\
\outline{Attack survey table}\\
\outline{Attack or current works about USB Type-C}\\

In this part, we survey related works on USB attack in Section~\ref{subsec:usb_attack}, and USB defence security in Section \ref{subsec:usb_defence}, respectively.

\subsection{USB Attack}
\label{subsec:usb_attack}
\chaozu{DoS(Fuzzing), Code injection, HID emulation, JFC, our work}
During the development of USB protocol, there are many USB-based attacks were proposed, ranging from DoS (denial of service) to protocol masquerading. Here we will introduce them following an order of their category.\hongyi{What?}

From the kernel perspective, its USB software stack generally expects devices to follow the USB standard and may not consider corner cases of malformed USB packets. Based on this, Facedancer\cite{facedancer} and Syzkaller\cite{syzkaller} uses fuzzing technique to uncover the bugs lying in the kernel drivers. These bugs can cause kernel crash and lead to a DoS situation. Though this poses a great challenge to the availability of a system, this attack still requires physical access to the host and is unable to cause more damage other than DoS.

In the field of USB security, protocol masquerading is also a widely used attack scheme. Due to the lack of authentication in USB protocol, malicious devices can hide their real functionality with re-written firmware\cite{rubber,badusb, rubberducky2020, usbbypassing, iseeyou, usbdriver}. These works rewrite the firmware of a normal-looking flash drive, which allows it to pretend as other devices. When these modified drives are connected to the host, they will be recognized as keyboard or mouse. Then the attacker can execute malicious payloads as they were using the victim's devices. But due to the limitation of USB 2.0\cite{usb20} protocol, these BadUSB attacks are unable to obtain video feedback from the victim as video were not supported until USB 3.0 \cite{usb30}. Though USB 2.0 does not support video transmission, there exists a protocol called MHL(Mobile High-Definition Link) which extends the USB standard and allow video signal to be transmitted through USB interface. JFC\cite{JFC}, short for juicy filming charging is such a work which abuses this standard and exfiltrate video data from victim without permission. But in JFC, this data exfiltration is not combined with BadUSB attacks and MHL is an outdated standard, which limits its capability.

Besides attacking from the protocol perspective, there are works trying to use USB device as a payload delivery means. Duqu\cite{duqu} uses a user-mode rootkit to hide malicious files on the USB storage device and uses a zero-day exploit\cite{zero-day} to execute the malware automatically. There are also works like \cite{brain, stuxnet, conficker,flame} following the same paradigm and performing code-injection attacks. These attacks are much more damaging and flexible comparing to those previous ones, but they requires certain existing flaw like \cite{zero-day} and USB are merely a payload delivery method.

As a data transmission protocol, USB inevitably leaks electromagnetic signals to the environment which may contains sensitive information. Leveraging this physical phenomenon, there are works like \cite{smartphone, poweremi,revealing,su2017usb, usbgpslocator, bates2014leveraging, badusbhub, usbfinger, side, usbdriver} eavesdropping leaked signals and recovering the sensitive data. In a similar fashion, \cite{usbkiller, cable, usbee, turnip} use the RF transmitter to inject signal to cause physical damage to the host machine. Even though the data, including the video data, could be recovered in this way, these attacks for executing malicious code is too difficult to work, and the invisibility is a problem cannot be ignored due to the spacial locality of radio frequency. 

Since USB 3.1 was introduced with USB Type-C in 2013, display port and HDMI connectors have been provided by USB Type-C, transferring of video data can be combined with the other attacks, like protocol masquerading,  protocol corruption and code injection. This has paved the path for our \tool.




\chaozu{attack table}\\
\chaozu{enum diagram}


\subsection{USB Defence Security}
\label{subsec:usb_defence}
There are already many defenses proposed against BadUSB attacks and a comprehensive investigation of previous work.\cite{sok}.

From the hardware perspective, BadUSB attack requires `D-' and `D+' pins which are defined by protocol to transmitting data.
Without these pins, data can't be transferred via USB cable. Based on this fact, USB Condom \cite{Condom} is a hardware solution to block data channel by adding blocker in the connector. This blocker can cut off the `D-' and `D+' connection while leave power pins intact.
However, this method poses a great challenge to plug-and-play property of USB, as once it is deployed, it will stop all USB functions other than charging. 

Under the premise of ensuring the full functionality of USB devices, there have been some works trying to improve the security during connection establishment.
Windows Defender ATP\cite{windenfenderwhite} maintains a whitelist of USB devices, only devices on the whitelist are allowed to communicate with the host. This prevents all potential attacks from untrusted devices. But this requires users to have a certain safety awareness and technical background to maintain a valid whitelist. For example, a naive user may add the USB device from unknown sources to his/her whitelist without precaution. There are designs can overcome this drawback. For instance, Mohammadmoradi et al\cite{mohammadmoradi2018making} proposes a strategy to generate such a whitelist automatically. This strategy first generates a unique fingerprint for each device based on its functionality. Then these fingerprints are used to maintain a safe and valid whitelist of USB devices. There is another work mediating USB connectivity for Industrial Control. TMSUI\cite{yang2015tmsui} relies on rich experience of administrators to build a whitelist. However, some modified USB devices may hide their real functionality from the user.

To solve this flaw, GoodUSB\cite{tian2015defending} will report the functionality claimed by the device to user and let user decides whether to authorize. When a device is plugged in, GoodUSB will load its driver but limit its functionality until a series of authorizations is completed. These authorizations are designed to be performed manually. As BadUSB is normally unable to obtain the video stream of the host, it is impossible for an attacker to complete these authorization with automatic script. Thus this defense is sufficient for normal BadUSB attack. But if the attacker has the access to victim's screen, GoodUSB will be bypassed easily as the attacker can just complete these authorization manually and perform subsequent attacks.

After the USB enumeration and driver loading, there are also works trying to archive `defense in depth' against BadUSB attack.
Neuner et al.\cite{neuner2018usblock} prevents BadUSB attacks from malicious flash drive by analyzing the temporal characteristics of BadUSB-like attacks. This defense mechanism is effective because the attacker cannot obtain the screen of the host using just BadUSB. In this case, the malicious device can only inject key stoke in a very short time to reduce the risk of being discovered. This defect causes the typing characteristics of BadUSB to be detectable.  Pham et al. \cite{pham2010optimizing} optimizing windows security features. It can block the execution of unsigned files, the installation of unsigned driver carried on portable media. What's more, in GoodUSB, a VM is deployed in host to as a honeypot to detect and stop malicious behavior of USB devices. 

In addition to injection attacks, data theft attack is also one of the focuses of academic community. As mentioned in Section~\ref{subsec:usb_attack}, there exist an attack called JFC (juicy film charging)\cite{JFC} which abuses the MHL standard to steal video stream from the victim. In order to mitigate this issue, Meng et al.\cite{JFC} proposed a statistical model using status like GPU/CPU usage to detect JFC attack.

To summarize, there exists a clear trade-off between the effectiveness and the plug-and-play property. Though hardware disabling solution like USB Condom archives the almost absolute security, the functionality of USB is scarified. Other solutions like GoodUSB or whitelist are either bypassable or insufficient under certain cases. 
Some vendors may sacrifice defensive capabilities to improve usability, which allows attackers to take advantage of.

We summarize the former effort in USB attack and defense in Table~\ref{table:attack_vs_defense}, which illustrates the effectiveness of each defense against various attacks including our \tool.
\newcommand{\circlefull}{\includegraphics[scale=0.025]{Figs/circle_full.png}}
\newcommand{\circlehalf}{\includegraphics[scale=0.025]{Figs/circle_half.png}}
\newcommand{\circleempty}{\includegraphics[scale=0.025]{Figs/circle_empty.png}}
\begin{table*}
	\centering
	\begin{tabular}{|c|c|c|c|c|c|c|c|c|}
		
		\hline
		\diagbox[width=1.46in] {Defence}{Attack} & Facedancer\cite{facedancer}, Syzkaller\cite{syzkaller} & \cite{rubber,badusb, rubberducky2020, usbbypassing, iseeyou, usbdriver} & JFC\cite{JFC}&Duqu\cite{duqu}& \cite{brain, stuxnet, conficker,flame}&\cite{smartphone, poweremi,revealing,su2017usb, usbgpslocator, bates2014leveraging, badusbhub, usbfinger, side, usbdriver} &\cite{usbkiller, cable, usbee, turnip}&Armory\\
		\hline 
		USB condom \cite{Condom}& \circlefull & \circlefull & \circlefull & \circlefull &\circlefull& \circlefull& \circlefull &\circlefull\\
		\hline 
		Windows Defender ATP\cite{windenfenderwhite}, Mohammadmoradi et al\cite{mohammadmoradi2018making}, TMSUI\cite{yang2015tmsui}& \circleempty & \circlehalf & \circlehalf & \circlehalf &\circlehalf& \circleempty& ???????? &\circlehalf\\
		\hline 
		GoodUSB\cite{tian2015defending}& \circlefull & \circlefull & \circlefull & \circlefull &\circlefull& \circleempty& \circlefull &\circleempty\\
		\hline 
		
		Neuner et al.\cite{neuner2018usblock}& \circleempty & \circlefull & \circleempty & \circleempty &\circleempty& \circleempty& \circleempty &\circleempty\\
		\hline 
		Pham et al. \cite{pham2010optimizing}& \circleempty & \circleempty & \circleempty & \circlefull &\circlefull& \circleempty& \circleempty &\circleempty\\
		\hline
		JFCGuard\cite{JFC}& \circleempty & \circleempty & \circlefull & \circleempty &\circleempty& \circleempty& \circleempty & ??? \\
			\hline
	\end{tabular}
	\linebreak
    \begin{tablenotes}
	\footnotesize
	\item[1] \circlefull means that the defense is effective
	\item[2] \circlehalf means that the defense is partial effective
	\item[3] \circleempty means that the defense is uneffective
	\end{tablenotes}
	\caption{Effectiveness of defense against different attacks}
	\label{table:attack_vs_defense}
\end{table*}





\section{\tool}
\label{sec:badusb}
\subsection{Motivation}
\noindent\outline{Limitation of Rubber Ducky}\\
\outline{Our functionality different mode listing?}\\
\hongyi{The following mode is to be further decided}
\outline{Automatic Scripting Mode}\\
\outline{Remote Control Mode}\\
\outline{OCR/QR Recognition Mode}\\
As we introduced in the Section~\ref{sec:related_work}, there are plenty of existing work \shuqing{Add citation} focusing on BadUSB attack. 
Many of these take advantage of the \textit{trust-by-default} policy of PC, pretend normal HID devices and utilize USB protocol to perform attacks. 
However, these attacks suffer from various drawbacks:
\ding{182} attackers can only simulate limited types of devices such as HIDs (like keyboards and mice) and disks, which makes the attacks less effective;
\ding{183} accurate attacks couldn't be performed due to lack of user interface (UI).
Whatever HID the attackers simulate their USB devices to be, they couldn't obtain the UI to check the current situation, which makes it nearly impossible to locate their attacks precisely or confirm the effects after attacks.

In this work, we utilized the new features of USB 3.x \cite{usb31} \cite{usb32} to address the problems above.
Benefiting from latest protocol, we simulated video card and thus obtained the video stream to perform accurate attacks.

\subsection{Implementation}
\noindent\hongyi{The following mode is to be further decided}\\
\outline{DETAIL of each mode}\\
\outline{Automatic Scripting Mode}\\
\outline{Remote Control Mode}\\
\outline{OCR/QR Recognition Mode}\\
Though Rubber Ducky\cite{rubber} emulates as a HID device enabling various arbitrary execution attacks, fetching feedback from the victim is much more limited. Based on this restriction several defense mechanisms were proposed  like GoodUSB\cite{tian2015defending}\hongyi{We need more example}. These defense mechanism implemented a CAPTCHA-like\cite{captcha} procedure, which requires user to complete certain challenges when a unknown HID device is plugged in. As the Rubber Ducky is only able to emulate HID device, by no means the Rubber Ducky can bypass it. Our \tool, on the other hand, is capable of bypassing this type of defense. Taking advantage of USB 3.x\cite{usb31}\cite{usb32}, \tool can fetch the video stream of the victim and complete the challenge automatically or manually. Moreover, with video capability, our \tool achieved multiple new threat model and are tested under various scenarios.

As there were various Rubber Ducky implementation available, we only focus on its core functionality and our extension. Based on this pattern, we summaries our implementation of \tool into the following parts and their comparison can be found in Table~\ref{table:mode_comparison}:
\begin{table*}
	\centering
	\begin{tabular}{|c|c|c|c|c|}
	\hline
	Mode & Detectibility & Automation & Power Consumption  \\
	\hline
	Scripting & Medium & Yes & Low\\
	\hline
	Remote Control & High & No & High \\
	\hline
	Privacy Extraction & Low & Yes & Medium\\
	\hline
	\end{tabular}
	\linebreak
	\caption{Comparison of Different Mode}
	\label{table:mode_comparison}
\end{table*}
\begin{itemize}
	\item\textit{Hardware specification} the  hardware we uses to build \tool
	\item\textit{Scripting mode} functions like the original Rubber Ducky featuring the bypass of defense mechanism like GoodUSB\cite{tian2015defending}
	\shuqing{Cannot bypass USBCheckIn.}
	\item \textit{Remote control mode} allows attacker to take complete control of victim's device manually to perform delicate operation.
	\item \textit{Privacy extraction mode} captures and identify private data from the victim and send to the attacker.
\end{itemize}

\subsubsection{Hardware Specification}
The hardware we use is as follows and the relation between them is illustrated at Fig.~\ref{fig:hardware_specification}
\begin{itemize}
	\item\textit{USB 3.x Hub} exports the USB 3.x connector into various ports, like DisplayPort, USB 2.0 port etc.
	\item\textit{Video Capture Card} convert DisplayPort signal into UVC-compatible\hongyi{citation needed} data, which is later processed by Raspberry Pi.
	\item\textit{Raspberry Pi} processes video stream from the victim where private data is extracted and transmitted through GSM/WiFi.
	\item\textit{Rubber Ducky} emulates HID device which can be controlled from either the embedded Raspberry Pi or the attacker.
\end{itemize}
\begin{figure}[hbtp]
	\includegraphics[width=\linewidth]{./Figs/hardware_specification.png}
	\caption{Hardware Specification}
	\label{fig:hardware_specification}
\end{figure}
\subsubsection{Scripting Mode}
\tool under this mode works almost the same like the original Rubber Ducky with various improvement.

First \tool supports defense mechanism bypass. As the original Rubber Ducky cannot get feedback from the victim, defense mechanism like GoodUSB \cite{tian2015defending} is sufficient to prevent these BadUSB attack. But such defense mechanism relies on visual notification to request authentication for unknown USB device. This means that when the defense request authentication from the user, our \tool is able to capture the content of authentication challenge and complete it automatically or manually. After the defense is bypassed, \tool continues the emulation and script execution, resulting a successful attack.

As the mouse relies on the visual feedback to work properly, its emulation and automation were not supported by the original version of Rubber Ducky. Yet with the video output support from USB 3.x, our \tool implements full-functional mouse emulation. This function enables attacks toward pure GUI programs and showed great potential in mobile attack scenario. Details can be obtained in Section~\ref{sec:experiment}.
\subsubsection{Remote Control Mode}
In \tool, we implement all components required to operate a computer/mobile remotely, including a video stream, keyboard/mouse emulation.

\tool under this mode follows a simple logic. \tool will receive video stream from the victim's device and redirect it to the attacker via GSM/WiFi. In the meanwhile, \tool also receives commands from the attacker through GSM/WiFi and replay them to the victim by USB emulation.

This mode enable attacker to perform delicate operation that is beyond automation. Moreover, this mode provides a backdoor that does not require host network and thus is undetectable by firewall running on the host machine.
\hongyi{I think this is a quite good selling point?}
\subsubsection{Privacy Extraction Mode}
\tool under this mode does not emulate other USB device and solely rely on the video stream function of USB 3.x.

When running in this mode, \tool passively process the victim's video stream and detect for ``valuable'' private data.  Here, a data is ``valuable'' or not is decided by a customized detector, we implement a simple one of payment code for demonstration purpose. Though this detector is simple to implement, we successfully transfer money from the victim. More detail can be obtained in Section~\ref{sec:experiment}.

It is worth mentioning that \tool under this mode is completely passive, making it hard to be detected and defended. With different detector implemented, \tool under this mode is capable of serving more purpose.

\section{Experiment}
\label{sec:experiment}
\shuqing{Many parts in this section are much more likely to lie in implementation, introducing how \tool works. Doesn't look like experiment.}

%\shuqing{Experiment for devices and case study needed.}
To evaluate the effectiveness of \tool in different modes, we conducted three experiments on \tool using devices with USB Type-C capabilities from different OEMs, including a mobile phone, a tablet and a laptop.

\textit{Setup.} As mentioned in Section~\ref{sec:badusb}, our \tool only requires common components that are easily accessible online or in any electronic store. Here we choose the following parts to build a prototype. To begin with, we choose the Raspberry Pi 4B as the embedded computer inside \tool, which is powerful enough to process video data and has onboard WiFi chip. As for the HID emulator, we decide to use a ATMEGA32U4 board with USB protocol support. This Atmel chip is able to emulate most HID devices with our modified firmware. About the USB 3.x hub, we uses one from the Yamazawa, which supports HDMI, USB 2.0, and many other exported peripheral. Apart from these essential parts, we also uses an auxiliary power-bank to provide power for the Raspberry Pi and the mobile devices used by the victim. The image of our prototype \tool can be found in Figure~\ref{fig:armory}.
\shuqing{Figures of our power bank.}

\subsection{Scripting mode}
In the experiment of scripting mode, we used \textit{Lenovo Xiaoxin Pro 13 2020}, a PC in Windows 10/Ubuntu OS with two USB Type-C interfaces as target device. 
During this experiment, \tool disguised itself as a normal keyboard and a home-brewed version of GoodUSB\cite{tian2015defending} is deployed to test the defense bypassing. We have tried to deploy the original GoodUSB on the target device, but as the GoodUSB is proposed in 2015 and the code is too outdated to run at modern OS. We have to home-brew a similar defense mechanism according to its paper.
At the beginning, our home-brewed version of GoodUSB asked victim to manually complete a CAPTCHA to  authorize the new USB device, a.k.a our \tool. But at this time, by the design of GoodUSB, our \tool can already perform limited keystroke to complete the CAPTCHA and will be rejected only after the failure of authorization. With the video feedback from \tool, the attacker successfully completes the CAPTCHA.
Up to this point, the bypass of defense like GoodUSB is completed as our \tool already gain full function from the GoodUSB and become a trusted device. Then, our \tool works similar to the original BadUSB, inserting keystroke to perform arbitrary execution on the laptop. We tested three scripts, ranging from reverse shell backdoor to malware payload execution, all of which resulting in success.

With this experiment, we have proven the defense bypassing is possible using \tool. This also warned us that we need more thorough defense against such BadUSB attack.

\subsection{Remote control mode}
To test the capability of remote control mode, we chose \textit{iPad Pro (3rd generation)}, a tablet in iOS 14.3 with a USB Type-C interface, as the target device in this experiment.
Besides disguising as normal HID devices like a conventional BadUSB~\cite{badusb}, \tool also transmitted real-time video stream from the target device to the attacker via WiFi.
After the connection establishment, attacker performed a series actions to test the capability of \tool. At the beginning, attacker accessed album application on the iPad and obtained all the photos inside. After that, attacker tried to send messages via victim's account. At last, attacker performed a small amount of transaction using the financial application. All of these tests resulted in success and proved that \tool is a powerful attack tool.

Through this experiment, we have found that with video transmission and mouse emulation, \tool extensively expanded the attack surface of BadUSB attack, especially in mobile devices. We have archived complete hijack of victim's device in this experiment.

\subsection{Privacy extraction mode}
During the experiment of privacy extraction, we chose \textit{HUAWEI P30}, a smartphone in EMUI 9.1 (Android 9.0 based) with a USB Type-C interface, as the target device.
In privacy extraction mode, \tool passively captured video from the victim's device and used \textit{OpenCV} to identify valuable information from video stream. 
When the victim viewed text or photos with text, \tool used the techniques of optical character recognition (OCR) to extract text from corresponding video frames.
In this experiment, attacker had successfully extracted text like name, address, ID number and other valuable personal information. We also tested the payment code extractor, which enables attacker to identify payment code in the video stream and perform transaction without password. As this is also a part of our case study, more details about the data extracted can be found in Section~\ref{subsec:case_study} and Table~\ref{table:information_extracted}.

After the experiment, we have found that as privacy extraction mode only passively process victim's video stream, this is a more power-efficient mode than remote control mode.

\subsection{Case study}
\label{subsec:case_study}
\subsubsection{Background}

We conducted a case study with sharing power bank and QR code payment as technical background.

\textbf{Sharing Power Bank}. 
Sharing power bank provides users with short-term rental of power banks. 
The provider deploys power bank stations in the city, while the users can rent a power bank from any of the power bank station, charge their device on the trip, return the rented power bank to another station, and pay the rent online.

As an example, Brick is such a power bank sharing service provider from Sweden. 
As Brick's website states, \textit{Brick prevents your electronics from running out of battery with power banks (Bricks) that you can easily rent \& return at our many stations in Stockholm, Gothenburg, Malmö and the rest of Sweden and even Europe.}. \yechang{citation needed}
\shuqing{I think we should paraphrase instead of using the descriptions on the website directly. Just to explain \textbf{what we need}.}

\begin{figure}[t]
	\centering
	\includegraphics[width=.4 \linewidth]{./Figs/Brick_station.png}
	\includegraphics[width=.4 \linewidth]{./Figs/jiedian.jpg}
	\caption{Two power banks station products}\shuqing{Photos that we took.}
	\label{fig:PBS_products}
\end{figure}

\yechang{add hyperlink or reference to Brick's website or Brick App on App Store/Google Play.}

\yechang{Add more examples of power bank sharing to show that it is widely used?}
\shuqing{May use statistics (instead of concrete examples) to explain it.}

Though it provides convenience to users, it also brings security issues. 
We noticed that most of the power bank stations do not check the integrity of power banks during the rental process, and users are hardly cautious to check the power banks when connecting their devices to them. 
The attackers are able to modify their rented power banks, return it to a power bank station as normal.
Then the subsequent users will unknowingly connect to such malicious power banks, which endangers the financial security and data security of users.


\textbf{QR Code Payment}. 
QR code payment is a method of paying bills, typically performed as the following steps:
\ding{182} The payer opens the payment application on his/her mobile device, and present his/her payment QR code to the payee. 
This QR code is encoded with a globally unique ID to identify the payer's account. 
\ding{183} The payee scan the payment QR code presented by the payer. 
By presenting this QR code, the payer authorizes proceeding with the payment.
\ding{184} An order generated by this scan is sent to the payment service provider. 
Then the payment service provider requests the payer to confirm the transaction.
\ding{185} After confirmation, the payment service provider proceed with this transaction and return the payment result to both the payee and the payer.
\shuqing{I think this paragraph can be shorter. No need to explain so many details.}

\begin{figure}[t]
	\centering
	\includegraphics[width=\linewidth]{./Figs/qr_code_payment.png}
	\caption{Bar/QR code payment procedure}
	\label{fig:qr_payment_procedure}
\end{figure}


In the real-world scenario, some payment service providers provide special rules for micropayment purchases. 
A micropayment is pre-determined by the payment service provider with thresholds in the user agreement. 
For example, WeChat Pay~\cite{Wechat-pay} regards payments less than CNY \textyen 1000 as micropayments. 
Different from a typical payment procedure, when micropayments are made, confirmations can be applied automatically without requiring the payer to take further actions, which is often encouraged by the payment service provider.
The payment QR code is associated with the payer's account. 
If a victim's payment code is leaked to the attackers, they can use the that payment code to proceed payments. 
Furthermore, the attackers can use several micropayments to steal money from the victim's account, leaving the victim unconscious. 
\shuqing{Introduce something about QR code refreshing (fast).}
In summary, the payment QR code is highly sensitive on users' devices.

\subsubsection{Attack Scenario}

Note that \tool only exposes a type-c cable (from USB Type-C hub).
It charges the attacker's device through this cable, which has no difference from ordinary power banks. 
Considering users' vigilance and high similarity between \tool and normal power banks, the attackers can put the modified power banks (e.g., \tool) in a power bank station.
Subsequent users may unconsciously connect their devices to \tool.
Like many outdoor mobile phone usage scenarios\shuqing{Any reference? How frequent the payment code is used?}, the victim may complete payment with showing payment code. 
Then the attackers can extract and use the victim's payment code to make another payment unknown to the victim.
\shuqing{This paragraph is duplicate with some previous paragraphs. We can make the previous introduction abut attack scenario really brief, and mainly introduce it here.}

\shuqing{Background is longer than real user study.}
In order to validate the usability of \tool, we conducted a user study in privacy extraction mode. 
\shuqing{10 volunteers.}
We invited 6 volunteers to take part in, who are unconscious of the experiment details.
Volunteers took turns to operate their phones for half an hour while \tool is connected, which they considered as a normal power bank.
After the study, we introduced to them about our attack, checked the screen recordings together, and get the permission to study the information in recorded videos.
\shuqing{Do we need to mention that agreement the conference required here?}
Subject to efficiency reasons, we asked them to use phones just like when they use the shared power bank outside.
\shuqing{What does it mean?}


\subsubsection{Result}

After collecting the videos, we analyzed the videos both automatically and manually. 
During the automatic analysis, we used scripts to perform OCR recognition for each frame in the recorded video and stored all of the OCR results in a database.
At last, there are 38329 pieces of data collected among 6 volunteers.
\shuqing{The data may need to be updated.} 
With these results, we could learn what content the victim was browsing.
Additionally, some keywords such as \textit{account, username and password} often appear with users' input data because they are often used as labels of input boxes.
Such keywords are more likely to lead us to discover user-specific data.
For example, when we searched with \textit{account} as keyword, victim's accounts can be found in the database, as shown in the Table~\ref{tab:ocr_keyword_example}.
\shuqing{Statistics.}
The frame number is the position of this frame in the recorded video, which indicates a target for manual analysis for further data extraction.

\begin{table*}[t]
	\centering
	\begin{tabular}{|l|l|l|l|}
		\hline
		Keyword  & Text                                                                                                                          & Name                           & Frame Number \\ \hline
		username & X 8B cas.******.edu.cn Username: 117***18 Password:                                                                           & \textless{}user1\textgreater{} & 385          \\ \hline
		username & Login Weibo Login with SMS and verification code ...... +86 151****4587 Get verification code Login with username \& password & \textless{}user5\textgreater{} & 1947         \\ \hline
		username & QQ 14*****50| Login with phone number New user registration 2345678 9 0                                                       & \textless{}user3\textgreater{} & 4308         \\ \hline
		username & connect to *** username h*****l Save account information Open VPN.....                                                        & \textless{}user6\textgreater{} & 7925         \\ \hline
		+86      & Login with phone number ...... +86 186****2483 |                                                                              & \textless{}user1\textgreater{} & 313          \\ \hline
	\end{tabular}
	\caption{Example of searching OCR results with some keywords}
	\label{tab:ocr_keyword_example}
\end{table*}


In the manual analysis, we replayed the recorded video and extracted sensitive information.
The data we collected are listed in the Table~\ref{table:information_extracted}. 
Accounts of internet applications such as Apple, iCloud, Facebook, Twitter, etc. can be obtained. 
Moreover, all of the typing inputs on the virtual keyboard, including the system keyboard and the built-in security keyboard of the financial applications, can be clearly recorded.
We can obtain the plain text of password such as WiFi password.
Furthermore, the received SMS verification code (usually used to confirm real-name authentication) can be obtained when it appears in the top notification bar. 

In summary, though we can't directly obtain the user's password on the lock screen, we can still check all of the information presented on the screen, extract private information including but not limited to social accounts, bank accounts, personal financial situation, etc., if the user unconsciously unlocks the screen.
It is worth mentioning that, those \textit{secure keyboards} built in some financial apps just disrupt keyboard sequences, they can't prevent attacks similar as \tool.

\begin{table*}[t]
	\centering
	\begin{tabular}{|c|c|c|c|c|}
		\hline
		Application Column  & Application & Private information leaked                       \\
		\hline
		Finance App         & Alipay      & Alipay account, personal assets(blance)          \\
		\hline
		Social  Finance App & WeChat      & WeChat account, blance, chat history             \\
		\hline
		Social App          & QQ          & QQ account, interpersonal nexus, chat history    \\
		\hline
		Social App          & Twitter     & Twitter account, interpersonal nexus             \\
		\hline
		Social App          & Gmail       & Gmail account, mail records                      \\
		\hline
		Finance App         & ICBC        & ICBC account, password, personal assets(blance)  \\
		\hline
		Finance App         & Paypal      & Paypal account, blance, bank accounts            \\
		\hline
		Tool                & Chrome      & Sites visited                                    \\
		\hline
		Tool                & Health      & personal physical metrics      					 \\
		\hline
	\end{tabular}
	\linebreak
	\caption{Information extracted}\shuqing{Compress.}
	\label{table:information_extracted}
\end{table*}
\section{Countermeasures}
\label{sec:countermeasures}
Authorization mechanism like GoodUSB\cite{tian2015defending} has been proposed as a countermeasure against BadUSB attack. But as mentioned in Section~\ref{sec:badusb} and Section~\ref{sec:experiment}, GoodUSB relies on internal display to request for authorization, which can be hijacked by our \tool. Hence, GoodUSB is indeed a great defense against traditional BadUSB attack but no defense at all for our \tool. Here we discuss some effective countermeasures against \tool.

\textbf{External Hardware Authorization.}
One possible countermeasure is to introduce external hardware completing the authorization process. Contrary to the GoodUSB, USBCheckIn\cite{usbcheckin} adopts a dedicated hardware between the host and device. When a device is plugged in, the authorization will be complete on the dedicated hardware instead of internal display, preventing the host from being hijacked. Though USBCheckIn is a adequate defense against \tool, the external hardware brings additional cost and inconvenience, especially for mobile device.

\textbf{Isolated UI Rendering}
During our experiment, we noticed that \tool is actually unable to redirect out the locking screen keyboard from the iPad OS. Instead, the keyboard is only available on the internal display. However, this defense is only enabled on the locking screen keyboard, other virtual keyboard is still vulnerable to our \tool. This mechanism has inspired us to propose a new defense against our \tool. If the OS provider like Apple/Google can implement a more generalized \emph{isolated UI rendering} that allows developer to decide whether a content is ``sensitive'' and where it should be rendered. To better illustrate this countermeasure idea, we drew the Figure~\ref{fig:isolated_ui}. In that figure, the password keyboard is set to be ``sensitive'' component while the other parts of the UI is not. Thus the renderer will only render the password keyboard on the trusted internal display and prevent untrusted external display like \tool to obtain sensitive data. We believe this is a promising way to defend users from attack like juice filming and our \tool.
\begin{figure}[t]
	\centering
	\includegraphics[width=\linewidth]{./Figs/isolated_ui.png}
	\caption{Isolated UI Rendering}\shuqing{Maybe we can increase the font?}
	\label{fig:isolated_ui}
\end{figure}

\textbf{Distrust-by-Default.}
Most security issues of USB protocol is due to its \textit{trust-by-default} policy. \tool also relies on this feature to work. Hence if we reject all \textit{unauthorized} device, \tool and other many USB attacks will fail. It is worth mentioning here that \textit{distrust-by-default} policy is not the same as GoodUSB\cite{tian2015defending}. GoodUSB relies on the \textit{unauthorized} device to complete its own authorization, while in this strict policy, users have to use an \textit{authorized} device. Though \textit{distrust-by-default} policy effectively prevents these attack, this also causes considerable inconvenience for users. For example, when there is no other \textit{authorized} device plugged, it is impossible for user to complete the authorization in the first place. Thus this strict policy is far from optimal in most use cases.
\section{Discussion}
\label{sec:discussion}

\subsection{Limitation}
There exist multiple limitations of \tool. To begin with, when \tool is attached, the victim's device may not be set to mirror the screen. 
Depending on the victim's settings, the \tool may be used as an extended monitor or be set to `Desktop Mode' or not be enabled at all. Though we proposed the attack initialization in Section~\ref{subsec:attack_init} to inject keystrokes to ensure \tool is mirroring primary screen. But without knowing the exact operating system of the victim's device, \tool may have to try multiple injections before successfully mirroring.
Moreover, as mentioned in Section~\ref{sec:experiment}, \tool cannot directly obtain the information on the lock screen.
 Apart from that, we also noticed that in most devices, when an external monitor is connected to the device, there will be notifications about the monitor. In iPad, there is a small blue icon on the notification bar. In Windows 10, there is a pop-up for user to select the functionality of the external monitor. In latter case of a pop-up, \tool can dismiss it by injecting keystrokes as described in Section~\ref{subsec:attack_init}. But they are still noticeable by the victim. Also, it cannot be ignored that DisplayPort over \ac{USB} is not available on all devices. When selecting devices to test \tool, we find that many smartphones equipped with USB-C connectors actually do not support \ac{USB} 3.x protocol. There is a incomplete list of devices that support DisplayPort over \ac{USB}~\cite{usbclist}. Most vendors like HUAWEI and Samsung tend to support \ac{USB} 3.x protocol in their \mbox{high-end} smartphones. But there also exists vendor like Xiaomi who does not support \ac{USB} 3.x protocol at all. 

\begin{comment}
There exist multiple limitations of \tool.  To begin with, \tool can only gain
the information and control access of the host itself instead of external
hardware.  Consequently, as we introduced in the
Section~\ref{sec:countermeasures}, \tool can hardly bypass the defense
approaches that use external hardware for authorization.  Moreover, most of
the devices will prompt users to give authentication to the \ac{USB} devices or
select one of the functional modes after they are plugged in.  Though some of
such prompts are not conspicuous for non-experts, especially when \tool is
concealed within other functional hardware such as power banks \shuqing{If
there is experiment, add it here.}, the probability of whether users could get
aware that something unusual happens will increase with the existence of these
prompting messages.
\end{comment}
\subsection{Impact}

%\shuqing{Left after experiment to finish. Will discuss from different modes
%and application scenarios.}
The \ac{USB} protocol is used widely as introduced in the preceding sections.  As the
technologies develop rapidly, more and more devices will be equipped with USB-C
capabilities, which makes \tool more influential.  
Since non-professional users may neglect checking the security of plug-in \ac{USB} devices, 
they may unaware of the attacks from \tool.
can be hardly detected by them while attacks are performing.  
The popularity and universality of public \ac{USB} devices, including sharing power banks, even
increase such risks.  Moreover, \tool provides a better way for traditional
BadUSB attacks, since attackers can obtain the screen streaming with ease.
Attackers can use such technologies to perform more precise attacks, such as
interacting with the user interface and controlling the consequences of their
attacks.  In summary, \tool can be applied in various application scenarios and
brings rather huge impacts.

\section{Conclusion}
\label{sec:conclusion}
\outline{Conclusion}

Leveraging the new features of \ac{USB} 3.x~\cite{usb30,usb31,usb32}, we explore a
new attack scheme named \tool and three attack modes. Each of these three modes
has its own strength and use scenarios. The \ac{HID} emulation mode largely extends the
original BadUSB. The video capture mode proved practical and powerful, as it extracts
victim's sensitive information in a stealthy way. The full control mode
achieves the complete hijack of the target device, allowing us to perform various
types of subsequent attacks. By experimenting \tool with mobile phones, tablets,
and PCs, we further test its ability under different modes. In our experiment, we
obtain and analyze video stream extracted from 11 applications with \tool,
which demonstrates its capability in a simulated real-life scenario. In the end, we
propose a new defense scheme named \textit{isolated UI rendering}, which can
effectively stop attacks with \tool.
As for future work, we will further explore the potential of \textit{isolated UI
rendering} and implement it on a customized operating system. Moreover, we also
hope to lower the power consumption for network transmission in \tool to make
it more practical.

\section{Responsible Disclosure}
Responsible disclosure have already been carried out, we have contacted HUAWEI and Apple through proper channels. We received response from HUAWEI on March \nth{7} and discussed the mitigation plan on March \nth{10} while Apple has not responded yet. HUAWEI security team has confirmed this issue and been working on mitigation. We are actively facilitating their mitigation process and waiting for a fix to be deployed. After the mitigation is deployed, HUAWEI will assign a CVE ID for this vulnerability.

\section{Acknowledgement}
We are very grateful to our shepherd, Michael Schwarz, and the anonymous reviewers for their valuable feedback that improved the paper.

%\input{./Files/acknowledgement.tex}





\balance

\bibliographystyle{IEEEtran}
\bibliography{BadUSBC}


\end{document}
