\section{Background}
\label{sec:background}
%\shuqing{Maybe we could introduce HID devices somewhere.}
%\noindent\outline{USB Standard}\\

We first introduce the development of USB specification and emphasize the key
points adopted in this work. We also organized a brief timeline for introducing
key points of each protocol in Table~\ref{table:usb_timeline}.

%\noindent\outline{USB1.x}\\
Proposed in 1996, USB 1.0~\cite{usb10} was developed to provide a unified
interface and thus reducing the cost of reconfiguring the software. It is
worth mentioning that as a polled-bus interface, all data transfers are
initiated by the host.

%\noindent\outline{HID Protocol}\\
Right after one year of the appearance of USB 1.0, a standard named HID (Human
Interface Device)~\cite{hid} was born based on USB. HID is
designed with to unify the implementation for devices like mice,
keyboards, etc. Before its appearance, the standard is divided between
manufacturers, for example, the mouse of Company A may use X-Y coordinates to
represent its location while the mouse of Company B uses relative displacement.
This means every device needs its own driver to work. After HID, users only need to
write one driver for an entire class of HIDs. Furthermore, HID standard also
requires all devices to be PnP (\emph{Plug-and-Play}), which is indeed
convenient but insecure too.

In 1998, the first widely supported USB protocol was born. USB1.1~\cite{usb11}
provided two data transfer rates which are low speed (1.5 Mbit/s) and full
speed (12 MBit/s). At this point, due to the transfer limit, it only supports
limited types of devices like keyboards, mice, etc.

%\noindent\outline{USB2.0}\\
In 2000, USB 2.0~\cite{usb20} specification was released. With high speed \mbox{(480
Mbit/s)} mode introduced, printers, cameras, CD-ROM drives, and network cards were
supported in this revision. Such a high data transfer rate also gave rise to the
popularity of ``flash drive'', a portable device that allows physically
transferring data around~\cite{sok}. Though various peripherals were supported
in USB 2.0, there was no reliable way to identify the type of device. This
security flaw allowed attacks like BadUSB~\cite{badusb,rubber}.

%\noindent\outline{USB3.x}\\
USB 3.0~\cite{usb30} was announced in 2008, with a super speed (5 Gbit/s) data
transfer rate. Like its predecessor, more classes of peripherals were supported
in this revision. In 2013, the USB Type-C connector standard was introduced as a
part of USB 3.1~\cite{usb31}, providing a unified connector type for
PowerDelivery (PD), Thunderbolt, DisplayPort, and HDMI.  Yet no improvement of
security was introduced in 3.x revisions, meaning any device claiming
itself as a monitor can capture the video stream from the host. Exposing
such a multi-propose connector unprotected is insecure and allows attacks like
this work \tool. In 2017, USB 3.2~\cite{usb32} was released, doubling the data
transfer rate (20 Gbit/s).

%\noindent\outline{Connector Standard}
\begin{figure}[t]
    \centering
	\includegraphics[width=0.7\linewidth]{./Figs/usb_conn.png}
	\caption{USB 1.x \& 2.x Connector.}
	\label{fig:usb_conn}
\end{figure}

As illustrated in Figure~\ref{fig:usb_conn}, the original USB 1.x \& 2.x
connector only has two pins for data transferring \mbox{(D+ \& D-)}, which has
significantly limited data transfer rate \mbox{(5 Gbits/s Max)} and cannot support for
peripherals like DisplayPort \mbox{(10.8 Gbit/s Min)}. Apart from that, support for
other peripherals also require dedicated transferring lane as their standards
are not compatible with USB in most cases.  

\begin{figure}[t] 
	\centering
	\includegraphics[width=\linewidth]{./Figs/usb_c_conn.png} 
	\caption{USB Type-C Connector.} 
	\label{fig:usb_c_conn} 
\end{figure}

Thus, to provide support towards a wider range of peripherals, a 24-pins
standard called USB Type-C~\cite{typec} is introduced in 2013 by USB-IF~\cite{usbif}. As it is designed to be double-sided, the number of actually usable
pins is halved. Nevertheless, this standard has largely enhanced the capability
of the USB 3.x protocol. As presented in Figure~\ref{fig:usb_c_conn}, Type-C added
two high-speed data lanes \mbox{(TRX1 \& TRX2)} and kept the original data lane \mbox{(D+ \&
D-)}. The added lanes are used exclusively to support peripherals like
DisplayPort while the kept data lane transfers USB packets.

%\noindent\outline{Security Problem}\\
During the development of USB specification, security was rarely considered. As
the USB-IF believes it is the duty of original equipment manufacturers \mbox{(OEMs)}
to decide whether security features should be implemented~\cite{usbsec}. But the divergent implementations give a chance for attacks like
BadUSB~\cite{rubber} and our \tool.

%\fengwei{FIXME: add a reference for each attack in the Table.}
%\fengwei{FIXME: The table is too wide, exceeds the max size.}
\begin{table*}
\begin{tabular}{|c|l|c|c|c|}
	\hline
	\textbf{Year} & \textbf{Protocol Version} & \textbf{Supported Peripherals} & \textbf{Transfer Speed} & \textbf{Attacks} \\
	\hline
	1996 & USB 1.x~\cite{usb10,usb11} & Keyboard, Mouse... & 1.5 Mbit/s or 12 Mbit/s & HID Emulating (BadUSB)~\cite{badusb} \\
	\hline
	2000 & USB 2.0~\cite{usb20} & Flash Drive, High-Definition Link, CD Driver... & 480 Mbit/s & Autorun Attack~\cite{duqu}, Juice Filming~\cite{JFC,JFCImpact} \\
	\hline
	2008 & USB 3.0~\cite{usb30} & / & 5 Gbit/s & / \\
	\hline
	2013 & USB 3.1~\cite{usb31} & HDMI, DisplayPort, ThunderBolt... & 10 Gbit/s & \tool \\
	\hline
	2017 & USB 3.2~\cite{usb32} & / & 20 Gbit/s & / \\
	\hline
\end{tabular}
	\linebreak
\caption{USB Protocol Timeline.}
\label{table:usb_timeline}
\end{table*}
