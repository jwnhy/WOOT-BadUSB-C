\section{\tool}
\label{sec:badusb}
\subsection{Motivation}
\noindent\outline{Limitation of Rubber Ducky}\\
\outline{Our functionality different mode listing?}\\
\hongyi{The following mode is to be further decided}
\outline{Automatic Scripting Mode}\\
\outline{Remote Control Mode}\\
\outline{OCR/QR Recognition Mode}\\
As we introduced in the Section~\ref{sec:related_work}, there are plenty of existing work \shuqing{Add citation} focusing on BadUSB attack. 
Many of these take advantage of the \textit{trust-by-default} policy of PC, pretend normal HID devices and utilize USB protocol to perform attacks. 
However, these attacks suffer from various drawbacks:
\ding{182} attackers can only simulate limited types of HIDs like keyboards, mice, and disks\hongyi{disk is not HID}, which makes the attacks less effective;
\ding{183} accurate attacks couldn't be performed due to lack of user interface (UI).
Whatever HID the attackers simulate their USB devices to be, they couldn't obtain the UI to check the current situation, which makes it nearly impossible to locate their attacks precisely or confirm the effects after attacks.

In this work, we utilized the new features of USB 3.x \cite{usb31} \cite{usb32} to address the problems above.
Benefiting from latest protocol, we simulated video card and thus obtained the video stream to perform accurate attacks.

\subsection{Implementation}
\noindent\hongyi{The following mode is to be further decided}\\
\outline{DETAIL of each mode}\\
\outline{Automatic Scripting Mode}\\
\outline{Remote Control Mode}\\
\outline{OCR/QR Recognition Mode}\\
Though Rubber Ducky\cite{rubber} emulates as a HID device enabling various arbitrary execution attacks, fetching feedback from the victim is more limited. Based on this restriction several defense mechanisms were proposed  like USBCheckIn\cite{usbcheckin}\hongyi{We need more example}. These defense mechanism implemented a CAPTCHA-like\cite{captcha} procedure, which requires user to complete certain challenges when a unknown HID device is plugged in. As the Rubber Ducky is only able to emulate HID device, by no means the Rubber Ducky can obtain the content of challenge and bypass it. Our \tool, on the other hand, is capable of bypassing this type of defense. Taking advantage of USB 3.x\cite{usb31}\cite{usb32}, \tool can fetch the video stream of the victim and complete the challenge automatically or manually.

As there were various Rubber Ducky implementation available, we only focus on its core functionality and our extension. Based on this pattern, we summaries our implementation of \tool into the following modes:

\begin{itemize}
\item\textit{Scripting mode} functions like the original Rubber Ducky featuring the bypass of defense mechanism like USBCheckIn\cite{usbcheckin}
\item \textit{Remote control mode} allows attacker to take complete control of victim's device manually to perform delicate operation.
\item \textit{Privacy extraction mode} captures and identify private data from the victim and send to the attacker.
\end{itemize}

\subsubsection{Scripting Mode}
\subsubsection{Remote Control Mode}
\subsubsection{Privacy Extraction Mode}
