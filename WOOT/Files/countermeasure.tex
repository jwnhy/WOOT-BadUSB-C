\section{Countermeasures}
\label{sec:countermeasures}
Authorization mechanism like GoodUSB\cite{tian2015defending} has been proposed as a countermeasure against BadUSB attack. But as mentioned in Section~\ref{sec:badusb} and Section~\ref{sec:experiment}, GoodUSB relies on internal display to request for authorization, which can be hijacked by our \tool. Hence, GoodUSB is indeed a great defense against traditional BadUSB attack but no defense at all for our \tool. Here we discuss some effective countermeasures against \tool.

\textbf{External Hardware Authorization}
One possible countermeasure is to introduce external hardware completing the authorization process. Contrary to the GoodUSB, USBCheckIn\cite{usbcheckin} adopts a dedicated hardware between the host and device. When a device is plugged in, the authorization will be complete on the dedicated hardware instead of internal display, preventing the host from being hijacked. Though USBCheckIn is a adequate defense against \tool, the external hardware brings additional cost and inconvenience, especially for mobile device.

\textbf{Detection Based on Processor Status}
Proposed by Meng, JFCGuard\cite{MENG2018252} is a detection mechanism of Juice Filming attack. This is based on the fact that video output will increase the usage of both GPU/CPU. As both Juice Filming attack and our \tool  highly relies on the video stream hijacking.  Thus, in a similar way, we can monitor the usage of processors and adapt a statistic model to tell whether the host device is under \tool attack. Comparing to the external hardware authorization plan, this does not require extra hardware and is more transparent to users. We have assessed this defense in Section~\ref{sec:experiment} and experiments proves this is a promising mechanism\hongyi{EXPERIMENT!!!!!!!!!!}.

\textbf{Distrust-by-Default}
Most security issues of USB protocol is due to its \textit{trust-by-default} policy. \tool also relies on this feature to work. Hence if we reject all \textit{unauthorized} device, \tool and other many USB attacks will fail. It is worth mentioning here that \textit{distrust-by-default} policy is not the same as GoodUSB\cite{tian2015defending}. The GoodUSB relies on the \textit{unauthorized} device to complete its own authorization, while in this strict policy, users have to use an \textit{authorized} device. Though \textit{distrust-by-default} policy effectively prevents these attack, this also causes considerable inconvenience for users. For example, when there is no other \textit{authorized} device plugged, it is impossible for user to complete the authorization in the first place. Thus this strict policy is far from optimal in most use cases.