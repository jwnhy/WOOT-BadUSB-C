\section{Countermeasures}
\label{sec:countermeasures}

%Authorization mechanisms like GoodUSB~\cite{tian2015defending} have been
%proposed as countermeasures against BadUSB attacks. As mentioned in
%Section~\ref{sec:badusb} and Section~\ref{sec:experiment}, GoodUSB relies on
%internal display for authorization, which can be bypassed via our \tool. 
%Hence, GoodUSB is indeed a great defense against traditional BadUSB attack but
%no defense at all for our \tool. 
Next, we discuss our recommended countermeasures against \tool.

\textbf{External Hardware Authorization.} One possible countermeasure is to
introduce external hardware completing the authorization process. For example,
 \mbox{USBCheckIn}~\cite{usbcheckin} adopts a dedicated hardware between
the host and device. When a device is plugged-in, the authorization will be
conducted on the dedicated hardware instead of the internal display, preventing
the host from being hijacked. Although \mbox{USBCheckIn} is an adequate defense against
\tool, the external hardware brings additional cost and inconvenience,
especially for mobile devices.

\textbf{Distrust-by-Default.} Most security issues of \ac{USB} protocol are due to
its \textit{trust-by-default} assumption; \tool also relies on this feature to work.
To defend against \tool and other USB-based attacks, we can simply reject all
{unauthorized} devices -- applying the \textit{distrust-by-default} policy. For example, 
in GoodUSB~\cite{tian2015defending}, all \ac{USB} devices are enabled with corresponding 
functionality only after being authorized by the user. 
Defense like USB Condom distrusts all types of \ac{USB} devices by blocking data channels and stopping all \ac{USB} functions other than charging.
Such defense methods effectively stop attacks 
like \tool.

\textbf{Isolated \ac{UI} Rendering.} During our experiments, we noticed that \tool
is actually unable to mirror the lock screen keyboard from the iPad
OS. Instead, the keyboard is only available on the internal display. However,
this defense is only enabled on the lock screen keyboard, other keyboards
(e.g., the virtual ones used by the apps) are still vulnerable to our \tool.
This mechanism has inspired us to propose a new defense against our \tool
called \emph{Isolated \ac{UI} Rendering}. As illustrated in
Figure~\ref{fig:isolated_ui}, we designed two separated \ac{UI} render layers and corresponding drivers, one
is secure the another is insecure. When an application requires to render, it can
pass the content along with a tag identifying whether the content is
``sensitive'' or not. If the content is tagged with ``sensitive'', then the OS
could only render it on the secure layer, which only shows the sensitive content
(e.g., a keyboard) on the trusted screen. For the rest of the rendering, it
would render on both the trusted and untrusted display (e.g., insensitive
contents).
For example, in Figure~\ref{fig:isolated_ui}, the ``password'' and ``keyboard'' are recognized as sensitive while other parts of content are insensitive. Thus, the ``password'' and ``keyboard'' are not rendered on untrusted screen.

\begin{figure}[t]
	\centering
	\includegraphics[width=\linewidth]{./Figs/isolated_ui.png}
	\caption{Isolated \ac{UI} Rendering}%\shuqing{Maybe we can increase the font?}
	\label{fig:isolated_ui}
\end{figure}
%\fengwei{what are untrusted and trusted screens? are they internal and external displays?}
%\hongyi{Fixed by using the same term and additional description}
