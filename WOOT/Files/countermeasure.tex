\section{Countermeasures}
\label{sec:countermeasures}
Authorization mechanism like GoodUSB\cite{tian2015defending} has been proposed as a countermeasure against BadUSB attack. But as mentioned in Section~\ref{sec:badusb} and Section~\ref{sec:experiment}, GoodUSB relies on internal display to request for authorization, which can be hijacked by our \tool. Hence, GoodUSB is indeed a great defense against traditional BadUSB attack but no defense at all for our \tool. Here we discuss some effective countermeasures against \tool.

\textbf{External Hardware Authorization.}
One possible countermeasure is to introduce external hardware completing the authorization process. Contrary to the GoodUSB, USBCheckIn\cite{usbcheckin} adopts a dedicated hardware between the host and device. When a device is plugged in, the authorization will be complete on the dedicated hardware instead of internal display, preventing the host from being hijacked. Though USBCheckIn is a adequate defense against \tool, the external hardware brings additional cost and inconvenience, especially for mobile device.

\textbf{Isolated UI Rendering}
During our experiment, we noticed that \tool is actually unable to redirect out the locking screen keyboard from the iPad OS. Instead, the keyboard is only available on the internal display. However, this defense is only enabled on the locking screen keyboard, other virtual keyboard is still vulnerable to our \tool. This mechanism has inspired us to propose a new defense against our \tool. If the OS provider like Apple/Google can implement a more generalized \emph{isolated UI rendering} that allows developer to decide whether a content is ``sensitive'' and where it should be rendered. To better illustrate this countermeasure idea, we drew the Figure~\ref{fig:isolated_ui}. In that figure, the password keyboard is set to be ``sensitive'' component while the other parts of the UI is not. Thus the renderer will only render the password keyboard on the trusted internal display and prevent untrusted external display like \tool to obtain sensitive data. We believe this is a promising way to defend users from attack like juice filming and our \tool.
\begin{figure}[t]
	\centering
	\includegraphics[width=\linewidth]{./Figs/isolated_ui.png}
	\caption{Isolated UI Rendering}\shuqing{Maybe we can increase the font?}
	\label{fig:isolated_ui}
\end{figure}

\textbf{Distrust-by-Default.}
Most security issues of USB protocol is due to its \textit{trust-by-default} policy. \tool also relies on this feature to work. Hence if we reject all \textit{unauthorized} device, \tool and other many USB attacks will fail. It is worth mentioning here that \textit{distrust-by-default} policy is not the same as GoodUSB\cite{tian2015defending}. GoodUSB relies on the \textit{unauthorized} device to complete its own authorization, while in this strict policy, users have to use an \textit{authorized} device. Though \textit{distrust-by-default} policy effectively prevents these attack, this also causes considerable inconvenience for users. For example, when there is no other \textit{authorized} device plugged, it is impossible for user to complete the authorization in the first place. Thus this strict policy is far from optimal in most use cases.