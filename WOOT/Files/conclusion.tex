\section{Conclusion}
\label{sec:conclusion}
\outline{Conclusion}

Leveraging the new features of USB 3.x~\cite{usb31,usb31,usb32}, we explore a
new attack scheme named \tool and three attack modes. Each of these three modes
has its own strength and use scenarios. The HID Emulation mode largely extends the
original BadUSB and successfully bypasses existing defenses. The video capture mode proved practical and powerful in the user study, as it extracts
victim's sensitive information in a stealthy way. The full control mode
archives the complete hijack of the target device, allowing us to perform various
types of subsequent attacks. By experimenting \tool with mobile phones, tablets,
and PCs, we further test its ability under different mode and proves it can
bypass defenses such as GoodUSB~\cite{tian2015defending}. In our user study, we
obtain and analyze video stream extracted from 10 participants with \tool,
which demonstrates its capability in a real-life scenario. In the end, we
propose a new defense scheme named \textit{isolated UI rendering}, which can
effectively stop our \tool.

As future work, we will further explore the potential of \textit{isolated UI
rendering} and implement it on a customized operating system. Moreover, we also
hope to lower the power consumption for network transmission in \tool to make
it more practical.

\section{Responsive Disclosure}

Responsive disclosure have already been carried out, we have already contacted the affected vendors through proper channel and are waiting for mitigation being deployed.