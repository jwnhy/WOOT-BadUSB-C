\section{Conclusion}
\label{sec:conclusion}
\outline{Conclusion}

Leveraging the new features of \ac{USB} 3.x~\cite{usb30,usb31,usb32}, we explore a
new attack scheme named \tool and three attack modes. Each of these three modes
has its own strength and use scenarios. The \ac{HID} emulation mode largely extends the
original BadUSB. The video capture mode proved practical and powerful, as it extracts
victim's sensitive information in a stealthy way. The full control mode
achieves the complete hijack of the target device, allowing us to perform various
types of subsequent attacks. By experimenting \tool with mobile phones, tablets,
and PCs, we further test its ability under different modes. In our experiment, we
obtain and analyze video stream extracted from 11 applications with \tool,
which demonstrates its capability in a simulated real-life scenario. In the end, we
propose a new defense scheme named \textit{isolated UI rendering}, which can
effectively stop attacks with \tool.
As for future work, we will further explore the potential of \textit{isolated UI
rendering} and implement it on a customized operating system. Moreover, we also
hope to lower the power consumption for network transmission in \tool to make
it more practical.

\section{Responsible Disclosure}
Responsible disclosure have already been carried out, we have contacted HUAWEI and Apple through proper channels. We received response from HUAWEI on March \nth{7} and discussed the mitigation plan on March \nth{10} while Apple has not responded yet. HUAWEI security team has confirmed this issue and been working on mitigation. We are actively facilitating their mitigation process and waiting for a fix to be deployed. After the mitigation is deployed, HUAWEI will assign a CVE ID for this vulnerability.

\section{Acknowledgement}
We are very grateful to our shepherd, Michael Schwarz, and the anonymous reviewers for their valuable feedback that improved the paper.
