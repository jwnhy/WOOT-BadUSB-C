\section{Introduction}
\label{sec:introduction}

The Universal Serial Bus (USB) protocol has become popular around the world since its appearance, as it provides a unified and easy-to-use approach for a large range of devices to communicate with each other.
From version 1.0 till now, USB specification has evolved rapidly and offered more and more functionalities.
Nowadays, devices with USB support are ubiquitous.

On the other side of the coin, the security of USB has caused serious problems.
The designers of USB protocol did not care much about security issues as they wanted to make a easy-to-use protocol.
There are more than four hundred vulnerabilities referencing USB on CVE list~\cite{website:CVE-list}.
As a result, many attackers exploit these vulnerabilities and the \textit{trust-by-default} characteristics of USB to conduct attacks, which puts the privacy and financial security of USB users in danger~\cite{sok}.

BadUSB is a well-known class of firmware attacks~\cite{badusb}.
These attacks are conducted through modification of the device firmware, which will disguise normal USB devices as other types of devices that are \textit{trust-by-default} by the hosts.
Typical simulated devices include HID (short for human interface devices, including keyboards, mice, etc.) and disks.
Utilizing BadUSB, attackers can pretend themselves as normal users, typing malicious code to victim computers, downloading and executing malicious scripts, and copying out private data from disks.
Such attacks can easily escape from traditional anti-virus software, since it is very hard to distinguish them from normal USB devices.

Despite the advantageous features of BadUSB, there exist several limitations as follows.
(1) Attackers can't conduct attacks precisely, which decreases the capabilities of BadUSB attacks.
When performing attacks on another host, the attackers can not obtain the current user interface (UI) status which limits them from taking subsequent moves.
For example, it is hard for attackers to locate specific functional UI patterns like buttons and links on victim computers by disguising USB devices as mice.
That explains why typical BadUSB attacks often only stay in command line, using commands to download malicious scripts for execution.
However, these attacks may be intercepted by anti-virus software or firewall due to the usage of host network.
(2) To our best knowledge, existing BadUSB attacks only utilize the features of USB 2.0.
The releasing of USB 3.0 makes USB more powerful, with a higher transmission rate for data and the support towards a larger range of peripherals including DisplayPort, HDMI, PowerDelivery, etc.
BadUSB attacks can become more effective with the help of USB 3.0.
(3) There have emerged multiple efficacious countermeasures after the appearance of BadUSB.
For example, GoodUSB offers a defence method by limiting the functions of USB devices to users' expectations~\cite{tian2015defending}.
It provides a graphical interface for user to describe the functionalities or roles of the USB device and reject any usage beyond the description.
\shuqing{Can't bypass USBCheckIn.}

In this work, we proposed our approach to address the limitations mentioned above and implemented a multi-mode attack model of USB, named \tool.
\tool extends BadUSB to support the features of USB Type-C.
Since USB Type-C is capable to transfer video stream data, \tool could obtain the information of victim's graphical interface during attacks.
Combining it with simulation of traditional HIDs like keyboards and mice, attackers can perform precise attacks.
We did experiments to verify that \tool could also bypass most of the countermeasures for BadUSB since they often rely on interaction with the graphical interface.
Moreover, we implemented multiple attacking modes of USB attacks based on our approach to verify its effectiveness, including scripting, remote control and privacy extraction.
To improve the efficiency and performance of \tool, we designed a filtering algorithm to preprocess the video data before network transmission.
After implementation, we conducted a series of experiments for each attack mode.
\shuqing{Case study, evaluation, notable results.}

We summarize our key contributions as follows:
\begin{itemize}
	\item To our best knowledge, this is the first work to utilize new features of USB Type-C.
	The combination of new support with conventional BadUSB makes attacks more precise and effective.
	\item Our approach can bypass most of the previous countermeasures of BadUSB.
	\item \shuqing{Experiment, evaluation, notable results.}
\end{itemize}

The rest of this paper is structured as follows:
\shuqing{Fill in after the structure is finalized.}














