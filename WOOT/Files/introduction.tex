\section{Introduction}
\label{sec:introduction}

The \ac{USB} protocol has become popular worldwide
since its appearance in 1996, as it provides a unified and easy-to-use approach for an
extensive range of devices to communicate with each other.  From version 1.0 till
now, \ac{USB} specification has evolved rapidly and offered more and more
functionalities.  Nowadays, devices with \ac{USB} support are ubiquitous.

Conversely, the security of \ac{USB} has caused severe problems.
Recent research of all USB specifications indicates that security has not been taken into consideration.~\cite{sok}  There are more than 400
vulnerabilities related to \ac{USB} on CVE list~\cite{website:CVE-list}.  As a
result, many attackers exploit these vulnerabilities and the
\textit{trust-by-default} characteristics of \ac{USB} to conduct attacks, which puts
the privacy and financial security of \ac{USB} users in danger~\cite{sok}.

BadUSB is a well-known class of firmware attacks~\cite{badusb}.  These attacks
are conducted by modifying the device firmware, which are disguised
 as ordinary \ac{USB} devices as other types of devices that are \textit{trust-by-default}
by the hosts.  Typically, simulated devices include \ac{HID}~\cite{hid} and disks.  Utilizing BadUSB,
attackers can pretend to be regular users, typing malicious commands to
victims' computers, downloading and executing malicious scripts, and copying out
private data from disks.  Such attacks can easily avoid detection by traditional
anti-virus software since it is hard to distinguish them from ordinary \ac{USB}
devices.

Despite the advantageous features of BadUSB, there exist several limitations as
follows.  (1) Attackers cannot conduct attacks precisely, which decreases the
capabilities of BadUSB attacks.  When performing attacks on another host, the
attackers cannot obtain the current \ac{UI} status, limiting
them from taking subsequent moves.  For example, it is hard for attackers to
locate specific functional \ac{UI} patterns such as buttons and links on victim's
computers by disguising \ac{USB} devices, e.g., mice.  This explains why typical
BadUSB attacks often only stay in the command line, using commands to download
malicious scripts for execution.  However, these attacks may be intercepted by
anti-virus software or firewall due to the host network usage.  (2) To
our best knowledge, existing BadUSB attacks only utilize the features of \ac{USB}
2.0.  The release of \ac{USB} 3.0 makes \ac{USB} more powerful, with a higher
transmission rate for data and the support towards a more extensive range of
peripherals such as DisplayPort, HDMI and PowerDelivery.  BadUSB attacks
can become more effective with the help of newly supported features in \ac{USB} 3.x.
(3) There have emerged multiple efficacious countermeasures after the
appearance of BadUSB.  For example, GoodUSB offers a defense method by limiting
the functions of \ac{USB} devices to users' expectations~\cite{tian2015defending}.
It provides a \ac{GUI} for users to describe the functionalities or
roles of the \ac{USB} device and reject any usage beyond the description.
%\shuqing{Can't bypass USBCheckIn.}

In this work, we addressed the limitations mentioned
above and implemented a multi-mode attack model of \ac{USB}, named \tool.  \tool
extends BadUSB to support the features of \ac{USB} Type-C.
Although many smartphones equipped with USB-C connectors do not support \ac{USB} 3.x protocol, such as products of Xiaomi, more and more vendors like HUAWEI and Samsung tend to support \ac{USB} 3.x protocol in their \mbox{high-end} smartphones~\cite{usbclist}.
Since \mbox{\ac{USB} Type-C} can transfer video stream data, \tool could obtain the information of the
victim's \ac{GUI} during attacks.  Combining it with the emulation
of traditional \acp{HID}, e.g., keyboards and mice, attackers are capable of
performing precise attacks.
%We did experiments to verify that \tool could also
%bypass some countermeasures for BadUSB since they often rely on interaction
%with the graphical interface.

Moreover, we implemented multiple attacking
modes of \ac{USB} attacks based on our approach to verify its effectiveness,
including \ac{HID} emulator mode, video capture mode, and full control mode  To improve the
efficiency and performance of \tool, we designed a filtering algorithm to
preprocess the video data before network transmission.
We conducted
a series of experiments for each attack mode as well as for different types of
devices, including smartphones, personal computers, and tablet computers, to
validate the usability of \tool.  We also conducted a case study for attacks in
sharing power banks, one of the application scenarios of \tool.
After the validation of our attack model, we
proposed several defense methods as countermeasures, including external
hardware authorization, distrust-by-default, etc.  It is worth noting that we
designed a method, called isolated \ac{UI} rendering, to separate the user interface
into sensitive and insensitive layers.  Only the insensitive layer's content would be passed to the insecure driver and thus rendered on the external display, protecting the sensitive layer's content.
%\shuqing{Case study, evaluation, notable results.}

We summarize our key contributions as follows:

\begin{itemize}

    \item To our best knowledge, this is the first work to utilize new features
	of \ac{USB} Type-C.  The combination of new support with conventional BadUSB
	makes attacks more precise and effective.

    %\item Our approach can bypass some previous countermeasures of BadUSB.

    \item We conducted a case study and multiple experiments to validate the
	usability and effectiveness of \tool.  We also proposed several
	countermeasures for our attack model, which are reasonable and
	insightful.
	%\fengwei{Add countermeasures, in particular the isolation design.}
	%\shuqing{Isolated UI rendering is added in the last paragraph.}
\end{itemize}

The rest of this paper is structured as follows.  Section~\ref{sec:background}
provides the background of \ac{USB} specification.  Section~\ref{sec:related_work}
introduces the existing works of \ac{USB} security from the aspects of attacking and
defense, respectively.  In Section~\ref{sec:badusb}, we present the threat model
and the overall implementation of \tool in three different modes.  The
experiments we conducted are featured in
Section~\ref{sec:experiment}.  We present some possible countermeasures of \tool
in Section~\ref{sec:countermeasures}.  The limits and impacts of our approach
are discussed in Section~\ref{sec:discussion}, and the conclusion lies in
Section~\ref{sec:conclusion}.
%\shuqing{Fill in after the structure is finalized.}














