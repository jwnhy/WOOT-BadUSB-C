\section{Related Work}
\label{sec:related_work}
In this part, we survey related works on USB attack security in Section III-A, and USB defence security in Section III-B, respectively.
\subsection{USB Attack Security}
USB 1.0\cite{usb01} was first introduced in 1996, which was designed to provide interfaces for disparate peripherals and reduce the complexity of both hardware design and software configuration\cite{sok}. USB 1.0 provided the data transfer for Low Speed(1.5 Mbit/s) and Full Speed( 12Mbit/s). Before USB 2.0 was released in 2000 and widely used, USB attacks are limited to human layer error attacks, \cite{se}, \cite{goverment}, \cite{atkvec}, \cite{ueerreallydo}, e.g, throwing a bad USB device on purpose, are in order to damage host through plugging in a peripheral by trust default. 

USB 2.0 protocol increased the speed of data transfer and it was also expended to include digit cameras. According to the protocol, the host recognises the peripherals through enumeration, and this gives attacked chance to cheat the host. BadUSB\cite{badusb} is going to attack host by modifying the devices' identify, re-flash the firmware to add more functionalities to the USB devices, for example, it can present malicious interfaces as a sample HID interfaces or work as a network adapter on a USB thumb driver. Other USB attacks, \cite{rubber}, \cite{usbdriver}, \cite{usbbypassing}, \cite{iseeyou} are also masquerading the USB protocol and provided illegal functionality interface to the host, taking advantage of the permissive trust model in USB whereby the host fully trust the devices connected. \cite{syzkaller} use the fact that the host's USB software stack generally expects devices with a standard USB protocol, with the help of fuzzing techniques using FaceDancer \cite{facedancer} and debuggers, it takes a number of vulnerabilities to the kernel-model. Furthermore, expect from the protocol related attack, the radio frequency based sensitive data stealing is another kind of attack, \cite{smartphone} , \cite{poweremi}, \cite{badusbhub}, \cite{usbfinger}, \cite{side}, \cite{usbdriver} work based on signal eavesdropping to attack during the data extraction, and \cite{usbkiller}, \cite{cable}, \cite{usbee}, \cite{turnip} use the RF transmitter to inject signal to cause physical damage to the host machine.    

As our best knowledge, attacks based on USB 3.0\cite{usb03} have not been discovered. In this work, we focus on attack based on USB 3.0 in the layer of transport\cite{sok}, which use the USB protocol to control the host or inject malicious command to the host connected with Type-C\cite{typec}, and streal the data trough it.

\subsection{USB Defence Security}
