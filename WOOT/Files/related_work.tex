\section{Related Work}
\label{sec:related_work}
%\hongyi{Move the stuff about USB protocol to Background}\\
%\outline{Attack based on USB 1.0, briefly}\\
%\outline{Attack based on USB 2.0, focus on works about application and transport layer}\\
%\outline{Attack survey table}\\
%\outline{Attack or current works about USB Type-C}\\

We survey related works on USB attack in Section~\ref{subsec:usb_attack} and
USB security defense in Section~\ref{subsec:usb_defence}, respectively.

\subsection{USB Attack}
\label{subsec:usb_attack}
\chaozu{DoS(Fuzzing), Code injection, HID emulation, JFC, our work}
\fengwei{Related work section needs to be improved. Many typos and grammar mistakes can be found.}

During the development of USB protocol, many USB-based attacks were proposed,
ranging from DoS (Denial of Service) to protocol masquerading.

%Attacks against USB kernel drivers
From the kernel perspective, its USB software stack generally expects devices
to follow the USB standard and may not consider corner cases of malformed USB
packets. Based on this, Facedancer~\cite{facedancer} and
Syzkaller~\cite{syzkaller} use fuzzing techniques to uncover the bugs lying in
the kernel drivers. These bugs can cause kernel crash and lead to a DoS attack.
Though this poses a great challenge to the availability of a system, this
attack still requires physical access to the host and is unable to cause more
damage other than DoS.

% USB protocol attacks
In the field of USB security, protocol masquerading is also a widely used
attack scheme. Due to the lack of authentication in the USB protocol, malicious
devices can hide their real functionality with re-written
firmware~\cite{rubber,badusb,rubberducky2020,usbbypassing,iseeyou,usbdriver}.
These works rewrite the firmware of a normal-looking flash drive, which allows
it to pretend as other devices. When these modified drives are connected to the
host, they could be recognized as keyboard or mouse. Then the attacker can
execute malicious payloads as they were using the victim's devices. Due to the
limitation of USB 2.0~\cite{usb20} protocol, these BadUSB
attacks\fengwei{reference?} are unable to obtain video feedback from the victim
as video were not supported until USB 3.0~\cite{usb30}. Though USB 2.0 does not
support video transmission, there exists a protocol called MHL (Mobile
High-Definition Link) which extends the USB standard and allow video signal to
be transmitted through USB interface. JFC~\cite{JFC}, short for juicy filming
charging is such a work which abuses this standard and exfiltrate video data
from victim without permission. But in JFC, this data exfiltration is not
combined with BadUSB attacks and MHL is an outdated standard, which limits its
capability.

% Using USB devices to inject code
Besides attacking from the protocol perspective, there are works trying to use
USB device as a payload delivery means. Duqu~\cite{duqu} uses a user-mode
rootkit to hide malicious files on the USB storage device, \cite{flame} uses a
zero-day exploit and malicious \textit{autorun.inf} to execute the malware
automatically. There are also works like \cite{brain, stuxnet, conficker}
following the same paradigm and performing code-injection attacks. These
attacks are much more damaging and flexible comparing to those previous ones,
but they requires certain existing flaw like \cite{zero-day} and USB is merely
a payload delivery method.

% Using USB for information leakage
As a data transmission protocol, USB inevitably leaks electromagnetic signals
to the environment which may contains sensitive information. Leveraging this
physical phenomenon, previous works~\cite{smartphone,
poweremi,revealing,su2017usb,usbgpslocator,bates2014leveraging,badusbhub,usbfinger,side,usbdriver}
eavesdrop leaked signals and recover the sensitive data. In a similar fashion,
\cite{usbee,turnip} emit electromagnetic emissions by data injection on the bus
with the connected USB devices as a RF transmitter and \cite{usbkiller,
cable}\fengwei{we don't need to have space between references} inject analog
power to cause physical damage to the host machine. Even though the data,
including the video data, could be recovered in this way, these attacks for
executing malicious code is too difficult to work, and the invisibility is a
problem cannot be ignored due to the spacial locality of radio frequency.

Since USB 3.1 was introduced with USB Type-C in 2013, display port and HDMI
connectors have been provided by USB Type-C, transferring of video data can be
combined with the other attacks, like protocol masquerading,  protocol
corruption and code injection. This has paved the path for our \tool.




\chaozu{attack table}\\
\chaozu{enum diagram}


\subsection{USB Defence Security}
\label{subsec:usb_defence}
Many defenses have been proposed to defend against BadUSB attacks~\cite{sok}.

From the hardware perspective, BadUSB attack requires `D-' and `D+' pins which
are defined by protocol to transmitting data.  Without these pins, data cannot
be transferred via USB cable. Based
on this fact, USB Condom~\cite{Condom} is a hardware solution to block data
channel by adding blocker in the connector. This blocker can cut off the `D-'
and `D+' connection while leave power pins intact.  However, this method poses
a great challenge to plug-and-play property of USB, as once it is deployed, it
stops all USB functions other than charging.

Under the premise of ensuring the full functionality of USB devices, some works
try to improve the security during connection establishment.  Windows Defender
ATP~\cite{windenfenderwhite} maintains a whitelist of USB devices, only devices
on the whitelist are allowed to communicate with the host. This prevents all
potential attacks from untrusted devices. But this requires users to have a
certain safety awareness and technical background to maintain a valid
whitelist. For example, a naive user may add the USB device from unknown
sources to his/her whitelist without precaution. There are designs can overcome
this drawback. For instance, Mohammadmoradi et
al.~\cite{mohammadmoradi2018making} propose a strategy to generate such a
whitelist automatically. This strategy first generates a unique fingerprint for
each device based on its functionality. Then these fingerprints are used to
maintain a safe and valid whitelist of USB devices. There is another work
mediating USB connectivity for Industrial Control. TMSUI~\cite{yang2015tmsui}
relies on rich experience of administrators to build a whitelist. However, some
modified USB devices may hide their real functionality from the user.

To solve this flaw, GoodUSB~\cite{tian2015defending} reports the functionality claimed
by the device to user and let user decides whether to authorize. When a device
is plugged in, GoodUSB will load its driver but limit its functionality until a
series of authorizations is completed. These authorizations are designed to be
performed manually. As BadUSB is normally unable to obtain the video stream of
the host, it is impossible for an attacker to complete these authorization with
automatic script. Thus this defense is sufficient for normal BadUSB attack. But
if the attacker has the access to victim's screen, GoodUSB will be bypassed
easily as the attacker can just complete these authorization manually and
perform subsequent attacks.

After the USB enumeration and driver loading, there are\fengwei{I feel too many
"there are" style sentences.} also works trying to archive `defense in depth'
against BadUSB attack.  Neuner et
al.~\cite{neuner2018usblock}\fengwei{Typicall, we add "~" to create a space for
each citation.} prevents BadUSB attacks from malicious flash drive by analyzing
the temporal characteristics of BadUSB-like attacks. This defense mechanism is
effective because the attacker cannot obtain the screen of the host using
BadUSB. In this case, the malicious device can only inject key stoke in a very
short time to reduce the risk of being discovered. This defect causes the
typing characteristics of BadUSB to be detectable. Pham et
al.~\cite{pham2010optimizing} optimizing windows security features. It can
block the execution of unsigned files, the installation of unsigned driver
carried on portable media. Moreover, in GoodUSB, a VM is deployed in host to as
a honeypot to detect and stop malicious behavior of USB devices.

In addition to injection attacks, data theft attack is also one of the focuses
of academic community. As mentioned in Section~\ref{subsec:usb_attack}, there
exist an attack called JFC (juicy film charging)~\cite{JFC} which abuses the MHL
standard to steal video stream from the victim. In order to mitigate this
issue, Meng et al.~\cite{meng2018252} proposed a statistical model using status
like GPU/CPU usage to detect JFC attack.

To summarize, there exists a clear trade-off between the effectiveness and the
plug-and-play property. Though hardware disabling solution like USB Condom
archives the almost absolute security, the functionality of USB is scarified.
Other solutions like GoodUSB or whitelist are either bypassable or insufficient
under certain cases.  Some vendors may sacrifice defensive capabilities to
improve usability, which allows attackers to take advantage of.

We summarize the former effort in USB attack and defense in
Table~\ref{table:attack_vs_defense}, which illustrates the effectiveness of
each defense against various attacks including our \tool.

\newcommand{\circlefull}{\includegraphics[scale=0.025]{Figs/circle_full.png}}
\newcommand{\circlehalf}{\includegraphics[scale=0.025]{Figs/circle_half.png}}
\newcommand{\circleempty}{\includegraphics[scale=0.025]{Figs/circle_empty.png}}
\begin{table*}
	\centering
	\begin{tabular}{|c|c|c|c|c|c|c|c|}

		\hline
		\diagbox[width=1.52in,height=0.4in] {Defence}{Attack} & \makecell*[c]{Facedancer~\cite{facedancer},\\ Syzkaller~\cite{syzkaller}} &\cite{rubber, badusb, rubberducky2020, usbbypassing, iseeyou, usbdriver} & JFC~\cite{JFC}&		\makecell{
			Duqu~\cite{duqu}, \\
			\cite{brain, stuxnet, conficker,flame}} &\cite{smartphone, poweremi,revealing,su2017usb, usbgpslocator, bates2014leveraging, badusbhub, usbfinger, side, usbdriver, usbee, turnip}&\cite{usbkiller, cable}& \tool \\
		\hline
		\makecell{USB condom~\cite{Condom}} & \makecell*[c]{\circlefull} & \circlefull & \circlefull &\circlefull& \circlefull&  \circleempty&\circlefull\\
		\hline
		\makecell{
			Windows Defender ATP~\cite{windenfenderwhite}, \\
			Mohammadmoradi et al.~\cite{mohammadmoradi2018making}, \\
			TMSUI~\cite{yang2015tmsui}
		}& \circleempty & \circlehalf & \circlehalf &\circlehalf& \circleempty&  \circleempty &\circlehalf\\

		\hline
		\makecell{GoodUSB~\cite{tian2015defending}} & \makecell*[c]{\circlehalf} & \circlefull & \circlefull &\circlefull& \circleempty&  \circleempty &\circleempty\\
		\hline

		\makecell{Neuner et al.~\cite{neuner2018usblock}} & \makecell*[c]{\circleempty} & \circlefull & \circleempty &\circleempty& \circleempty& \circleempty &\circleempty\\
		\hline
		\makecell{Pham et al.~\cite{pham2010optimizing}} & \makecell*[c]{\circleempty} & \circleempty & \circleempty &\circlefull& \circleempty& \circleempty &\circleempty\\
		\hline
		\makecell{JFCGuard~\cite{meng2018252}} & \makecell*[c]{\circleempty} & \circleempty & \circlefull &\circleempty&  \circleempty  & \circleempty& \circlefull \\
			\hline
	\end{tabular}
	\linebreak
    \begin{tablenotes}
	\footnotesize
	\item[1] \circlefull  \@ means that the defense is effective
	\item[2] \circlehalf \@ means that the defense is partial effective
	\item[3] \circleempty \@  means that the defense is uneffective
	\end{tablenotes}
	\caption{Effectiveness of Defense against Different Attacks}
	\label{table:attack_vs_defense}
\end{table*}




