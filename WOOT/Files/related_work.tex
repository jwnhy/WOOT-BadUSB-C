\section{Related Work}
\label{sec:related_work}
\hongyi{Move the stuff about USB protocol to Background}\\
\outline{Attack based on USB 1.0, briefly}\\
\outline{Attack based on USB 2.0, focus on works about application and transport layer}\\
\outline{Attack survey table}\\
\outline{Attack or current works about USB Type-C}\\

In this part, we survey related works on USB attack security in Section III-A, and USB defence security in Section III-B, respectively.

\subsection{USB Attack Security}
While many USB based  attacks have been developed, these attacks are all tied to the environment of the target devices. In addition, limited to the USB 2.0\cite{usb20} specification, there is no video transformation interface, the control of attack processing is impossibly handled by the attackers without additional processing signal received. When the  protocol masquerading attack devices, e.g., Rubbery Ducky\cite{rubber}, \cite{rubberducky2020} and USBderivery\cite{usbdriver}, are connected to the host, they will hide their own functionality with the re-write functionality interface, that works because of the HID enumeration property for USB identifying. Software running on the host will exfiltrate data or receive commands via the RF interface in \cite{turnip} after modified the USB table with an additional RF transmitter. Due to the limitations of USB 2.0, it is impossible to handle the real-time screen information of the host, so only environment fitted commands or data will work in these scenarios.

In a similar way, \cite{duqu} used a user-mode rootkit to hide malicious files on the USB storage device, zero-day exploits is also developed in \cite{zero-day} to automatically execute malware when the storage devices are connected to the host. Due to the limitation of functionality interface of devices for video transformation, the accuracy timing for wake up the devices is difficult to handle, e.g., executing the malicious code in host's free time, pause the execution processing in the inappropriate time. 

Another kind of attacks for USB to exfiltrate data\cite{smartphone}\cite{poweremi} and \cite{usbdriver} is to eavesdrop signal to recovered the sensitive data at the physical layer. Even though the data, including the video data, could be recovered in this way, the attack for executing malicious code is difficult to work, and the invisibility is a problem cannot be ignored due to the spacial locality of radio frequency. 

Since USB 3.1 was introduced with USB Type-C in 2013, display port and HDMI connectors have been provided by USB Type-C, transformation for video streaming data is guaranteed with the other attacks, e.g., protocol masquerading,  protocol corruption and code injection. To our best knowledge, we are the first to combine the host interaction with attack based on USB. Based on USB Type-C, we design Armory, a multi-mode BadUSB device\chaozu{add armory description} 


\chaozu{attack table}\\
\chaozu{enum diagram}

\subsection{USB Defence Security}

Tian, Jing, et al.[1] have made a brief overview of the defense countermeasures over four layers attack in the article. For attacks against human-layer problems, [2],[3]raise people's awareness of social engineering,[][][4],[5],[6] prevent the data from being stolen by on-device encryption, [7],[8] provide host authentication, [8]monitor data flow to record potential malicious IO behaviors and [9],[10]block physical connections to defense. In application layer,modification over OS and driver are adapted. [11],[12],[13] harden  host system to prevent automatic execution of malicious code. GOOD USB[14] offers strict access control, with the system soliciting desired functionality from the user and blocking other functionality for a device. Ghost[15] simulates as a USB device to detect if the host has malicious code injection. In transport layer, 

FirmUSB[16] and VIPER[17] check the firmware of the device to ensure its safety. Fuzzing is a method of detecting vulnerabilities by providing unexpected input to the target system and monitoring for abnormal results. Some related work,[][][18],[19],[20] ,focuses on using USB fuzzing approach to detect potential bugs in USB driver. USBFirewall[21] protects the host based on error detection of USB communication packages. In some related work, [14],[22], e. g, VM is deployed in host to interact with the USB device for malicious behavior detection. In physical layer, some related work like [23] consider anti-fingerprinting ,and some others like [24] consider data encryption on USB bus.