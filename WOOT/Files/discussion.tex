\section{Discussion}
\label{sec:discussion}

\subsection{Limitation}
There exist multiple limitations of \tool. To begin with, when \tool is attached, the victim's device may not be set to mirror the screen. Depending on victim's settings, the \tool may be used as an extended monitor. For some smartphone vendors (e.g. Smartisan and Huawei), their devices are equipped with Desktop mode, which allows user to display an desktop on the external monitor. In these cases, the victim would easily notice the presence of \tool and become suspicious. Apart from that, we also noticed that in most devices, when an external monitor is connected to the device, there will be notifications about the monitor. In iPad, there is a small blue icon on the notification bar. In Huawei P30, there is a pop-up for user to select the functionality of the external monitor. In latter case of a pop-up, \tool can dismiss it by injecting keystrokes and mouse movements. But they are notifications after all and victims may notice \tool is a malicious device. Also, it cannot be ignored that DisplayPort over USB / DisplayPort Alternate Mode is not available on all devices. When selecting our experiment devices, we find that many smartphones equipped with USB-C connector does not support USB 3.2 protocol. This lack of support further limited \tool. But we believe that USB 3.2 will be more popular and better supported, so it is important for us to take defense against attacks like \tool in advance.
\begin{comment}
There exist multiple limitations of \tool.  To begin with, \tool can only gain
the information and control access of the host itself instead of external
hardware.  Consequently, as we introduced in the
Section~\ref{sec:countermeasures}, \tool can hardly bypass the defense
approaches that use external hardware for authorization.  Moreover, most of
the devices will prompt users to give authentication to the USB devices or
select one of the functional modes after they are plugged in.  Though some of
such prompts are not conspicuous for non-experts, especially when \tool is
concealed within other functional hardware such as power banks \shuqing{If
there is experiment, add it here.}, the probability of whether users could get
aware that something unusual happens will increase with the existence of these
prompting messages.
\end{comment}
\subsection{Impact}

%\shuqing{Left after experiment to finish. Will discuss from different modes
%and application scenarios.}
The USB protocol is used widely as introduced in the preceding sections.  As the
technologies develop rapidly, more and more devices will be equipped with USB-C
capabilities, which makes \tool more influential.  Due to the neglect to
check the security of plugged-in USB devices of non-professional users, \tool
can be hardly detected by them while attacks are performing.  The popularity
and universality of public USB devices, including sharing power banks, even
increase such risks.  Moreover, \tool provides a better way for traditional
BadUSB attacks, since attackers can obtain the screen streaming with ease.
Attackers can use such technologies to perform more precise attacks, such as
interacting with the user interface, as well as control the consequences of their
attacks.  In summary, \tool can be applied in various application scenarios and
brings rather huge impacts.
