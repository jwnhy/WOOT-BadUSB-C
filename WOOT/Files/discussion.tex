\section{Discussion}
\label{sec:discussion}

\subsection{Limitation}

There exist multiple limitations of \tool.  To begin with, \tool can only gain
the information and control access of the host itself instead of external
hardware.  Consequently, as we introduced in the
Section~\ref{sec:countermeasures}, \tool can hardly bypass the defense
approaches which use external hardware for authorization.  Moreover, most of
the devices will prompt users to give authentication to the USB devices or
select one of the functional modes after they are plugged in.  Though some of
such prompts are not conspicuous for non-experts, especially when \tool is
concealed within other functional hardware such as power banks \shuqing{If
there is experiment, add it here.}, the probability whether users could get
aware that something unusual happens will increase with the existence these
prompting messages.

\subsection{Impact}

%\shuqing{Left after experiment to finish. Will discuss from different modes
%and application scenarios.}
The USB protocol is used widely as introduced in preceding sections.  As the
technologies develop rapidly, more and more devices will be equipped with USB-C
capabilities, which makes \tool more influential.  Due to the neglection to
check the security of plugged-in USB devices of non-professional users, \tool
can be hardly detected by them while attacks are performing.  The popularity
and universality of public USB devices, including sharing power banks, even
increase such risks.  Moreover, \tool provides a better way for traditional
BadUSB attacks, since attackers can obtain the screen streaming with ease.
Attackers can use such technologies to perform more precise attacks, such as
interacting with user interface, as well as control the consequences of their
attacks.  In summary, \tool can be applied in various application scenarios and
brings rather huge impacts.
