\section{Discussion}
\label{sec:discussion}

\subsection{Limitation}
There exist multiple limitations of \tool. To begin with, when \tool is attached, the victim's device may not be set to mirror the screen. 
Depending on the victim's settings, the \tool may be used as an extended monitor or not be enabled at all. Additionally, some smartphone vendors (e.g. Huawei and Smartisan) provide a Desktop Mode to users, which displays a desktop on external screen instead of mirroring. 
Moreover, as mentioned in Section~\ref{sec:experiment}, \tool cannot directly obtain the information on the lock screen.
In these cases, \tool is either easily noticeable to the victim or disabled.  Apart from that, we also noticed that in most devices, when an external monitor is connected to the device, there will be notifications about the monitor. In iPad, there is a small blue icon on the notification bar. In Windows 10, there is a pop-up for user to select the functionality of the external monitor. In latter case of a pop-up, \tool can dismiss it by injecting keystrokes as described in Section~\ref{subsec:attack_init}. But they are still noticeable by the victim. Also, it cannot be ignored that DisplayPort over \ac{USB} is not available on all devices. When selecting devices to test \tool, we find that many smartphones equipped with USB-C connectors actually do not support \ac{USB} 3.x protocol. There is a incomplete list of devices that support DisplayPort over \ac{USB}~\cite{usbclist} Most vendors like Huawei and Samsung tend to support \ac{USB} 3.x protocol in their \mbox{high-end} smartphones. But there also exists vendor like Xiaomi who does not support \ac{USB} 3.x protocol at all. 

\begin{comment}
There exist multiple limitations of \tool.  To begin with, \tool can only gain
the information and control access of the host itself instead of external
hardware.  Consequently, as we introduced in the
Section~\ref{sec:countermeasures}, \tool can hardly bypass the defense
approaches that use external hardware for authorization.  Moreover, most of
the devices will prompt users to give authentication to the \ac{USB} devices or
select one of the functional modes after they are plugged in.  Though some of
such prompts are not conspicuous for non-experts, especially when \tool is
concealed within other functional hardware such as power banks \shuqing{If
there is experiment, add it here.}, the probability of whether users could get
aware that something unusual happens will increase with the existence of these
prompting messages.
\end{comment}
\subsection{Impact}

%\shuqing{Left after experiment to finish. Will discuss from different modes
%and application scenarios.}
The \ac{USB} protocol is used widely as introduced in the preceding sections.  As the
technologies develop rapidly, more and more devices will be equipped with USB-C
capabilities, which makes \tool more influential.  
Since non-professional users often neglect checking the security of plug-in \ac{USB} devices, they are often unaware of the attacks from \tool.
can be hardly detected by them while attacks are performing.  
The popularity and universality of public \ac{USB} devices, including sharing power banks, even
increase such risks.  Moreover, \tool provides a better way for traditional
BadUSB attacks, since attackers can obtain the screen streaming with ease.
Attackers can use such technologies to perform more precise attacks, such as
interacting with the user interface and controlling the consequences of their
attacks.  In summary, \tool can be applied in various application scenarios and
brings rather huge impacts.
