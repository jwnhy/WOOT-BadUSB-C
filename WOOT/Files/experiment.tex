\section{Experiment}
\label{sec:experiment}
\shuqing{Many parts in this section are much more likely to lie in implementation, introducing how \tool works. Doesn't look like experiment.}

%\shuqing{Experiment for devices and case study needed.}
To evaluate the effectiveness of \tool, we conducted experiments in the three modes introduced in the Section~\ref{sec:badusb} on devices with USB Type-C capabilities.

\textit{Setup.} Our implementation contains a USB Type-C hub with HDMI and USB port, a Raspberry Pi 4B with a Wi-Fi chip onboard, a video capture card adapting HDMI signal to USB camera, and a normal power bank with dual outputs.
\shuqing{Figures of our power bank.}
The power bank is used to supply the Raspberry Pi 4B and charge the victim's devices.
The USB Type-C hub adapts USB Type-C to a HDMI port and USB ports, which is used to connect HID devices and transmit video stream from victim's devices to HDMI port. 
The video stream will be captured by the Raspberry Pi 4B through the video capture card.
Note that in the following experiments, we chose different types of devices, i.e., smartphones, PCs and iPads, to conduct experiments in different modes to validate the usability and effectiveness of \tool.

\subsection{Scripting mode}

We used \textit{Lenovo Xiaoxin Pro 13 2020}, a PC in Windows 10 with two USB Type-C interfaces as target device while conducting the experiment in scripting mode. 
In this experiment, \tool disguised itself as a normal keyboard.
Similar as BadUSB~\cite{badusb}, \tool injected keystrokes at a high speed to start a terminal (cmd.exe) by hotkey, and executed the malicious scripts in it.
Through executing scripts, \tool could implant viruses, worms, backdoors, etc. into hosts, which would make the victim's devices under significant risk.
However, scripting mode can only be used on devices which terminal is available, limiting its capabilities to PC.

\subsection{Remote control mode}
We chose \textit{iPad Pro (3rd generation)}, a tablet in iOS 14.3 with a USB Type-C interface, as the target device in this experiment.
Besides performing as normal HID devices as conventional BadUSB~\cite{badusb}, \tool could transmit real-time video streaming of screen from the target device to the attacker through WiFi or GSM.
\tool captured video from USB Type-C Hub's HDMI output via a video capture card, streamed video with \textit{FFmpeg}, a video processing utility and uploaded video stream to the attacker's device.
With a real-time video stream as an indicator, attackers can precisely control the position of mouse movements and clicks, controlling the iPad to open applications, peek at the victim's photos, etc.

Remote control mode is ideal for the attackers to control the victim's devices and grab all information shown on the victim's devices. 
However, it needs user consent for the first time and requires high power consumption to encode video streams and stable network connections.
In the real world, high power consumption will increase the possibilities of being noticed by the victim, and stable network connections is hardly guaranteed.


\subsection{Privacy extraction mode}
We chose \textit{HUAWEI P30}, a smartphone in EMUI 9.1 (Android 9.0 based) with a USB Type-C interface, as the target device in this experiment.
As remote control mode, \tool captured video from the victim's device and use \textit{opencv} to identify valuable information from video stream. 
When the victim viewed text or photos with text, \tool used the techniques of optical character recognition (OCR) to extract text from corresponding video frames. \yechang{hyperlink to OCR?}\shuqing{I think we don't need to explain OCR nor put a reference link.}
For example, we extracted the victim's name, photo, address, and ID number when victim is viewing his/her photos of his/her ID card or passport documents,
We also extracted the victim's payment information when victim is using bank apps or payment apps.
Besides, \tool used \textit{opencv} to detect and decode QR codes or barcodes shown on the victim's device, which is expected to extract payment code information in the payment apps.
Here is an example of extracting payment code with opencv library in listing. \shuqing{Need to be modified.}\shuqing{Do we plan to add this?}

Privacy extraction mode only extracts valuable information for attackers (or highly sensitive information for victims), which is more efficient  than remote control mode.

%\begin{lstlisting}[caption={python script for extracting payment code of victim},label=lst2:qr]
%import cv2, qrcode,requests
%import pyzbar.pyzbar as pyzbar
%def decodeDisplay(video):
%    gray = cv2.cvtColor(video, cv2.COLOR_BGR2GRAY)
%    barcodes = pyzbar.decode(gray)
%    for barcode in barcodes:
%        barData = barcode.data.decode()
%        barType = barcode.type
%        requests.post('<server of attacker>',
%            data={
%                'data': barData,
%                'type': barType
%            }
%        )
%if __name__ == '__main__':
%    cam = cv2.VideoCapture(0)
%    while True:
%        ret, frame = cam.read()
%        decodeDisplay(frame)
%        if cv2.waitKey(5) == 27:
%            break
%    cam.release()
%    cv2.destroyAllWindows()
%\end{lstlisting}


\subsection{Case study}

\subsubsection{Background}

We conducted a case study with sharing power bank and QR code payment as technical background.

\textit{Sharing Power Bank}. 
Sharing power bank provides users with short-term rental of power banks. 
The provider deploys power bank stations in the city, while the users can rent a power bank from any of the power bank station, charge their device on the trip, return the rented power bank to another station, and pay the rent online.

As an example, Brick is such a power bank sharing service provider from Sweden. 
As Brick's website states, \textit{Brick prevents your electronics from running out of battery with power banks (Bricks) that you can easily rent \& return at our many stations in Stockholm, Gothenburg, Malmö and the rest of Sweden and even Europe.}. \yechang{citation needed}
\shuqing{I think we should paraphrase instead of using the descriptions on the website directly. Just to explain \textbf{what we need}.}

\begin{figure}[hbtp]
	\centering
	\includegraphics[width=.4 \linewidth]{./Figs/Brick_station.png}
	\includegraphics[width=.4 \linewidth]{./Figs/jiedian.jpg}
	\caption{Two power banks station products}\shuqing{Photos.}
	\label{fig:PBS_products}
\end{figure}

\yechang{add hyperlink or reference to Brick's website or Brick App on App Store/Google Play.}

\yechang{Add more examples of power bank sharing to show that it is widely used?}
\shuqing{May use statistics (instead of concrete examples) to explain it.}

Whiling providing convenience to users, it brings security issues. 
We notice that the power bank station does not check the integrity of power banks during the rental process, and users are hardly cautious when connecting their devices to the power bank. 
A malicious user is able to modify his/her rented power bank, return it to a power bank station as normal, and its next user will unknowingly connect to this malicious power bank, which endangers the privacy of user.


\textbf{\textit{QR Code Payment}}. 
QR code payment is a payment method where payment is typically performed as the following steps:
\ding{182} The payer opens the payment application on his/her mobile device, and present his/her payment QR code to the payee. This QR code is encoded with a globally unique ID to identify the payer's account. 
\ding{183} The payee scan the payment QR code presented by the payer. By presenting this QR code, the payer authorizes proceeding with the payment.
\ding{184} An order generated by this scan is sent to the payment service provider. Then the payment service provider requests the payer to confirm the transaction.
\ding{185} After confirmation, the payment service provider proceed with this transaction and return the payment result to both the payee and the payer.

\begin{figure}[hbtp]
	\centering
	\includegraphics[width=\linewidth]{./Figs/qr_code_payment.png}
	\caption{Bar/QR code payment procedure}
	\label{fig:qr_payment_procedure}
\end{figure}


In the real-world, some payment service providers provide special rules for micropayments purchases. A micropayment is pre-determined by the payment service provider with thresholds in the user agreement. For example, WeChat Pay regards payments less than CNY \textyen 1000 as micropayments. Differing from a typical payment procedure, when micropayments are made, confirmations can be applied automatically without requiring the payer to take further action, which is often encouraged by the payment service provider.

The payment QR code is associated with the payer's account. If a victim's payment code is leaked to an attacker, an attacker can use the victim's payment code to proceed with payments. Furthermore, the attacker can use several micropayments to steal money from the victim's account while leaving the victim unconscious. In summary, the payment QR code is highly sensitive on users' devices.

\subsubsection{Attack scenario}

It is worth noting that \tool exposed only a type-c cable (from USB Type-C hub),
and \tool will charge his or her device through this cable, which has no difference from ordinary power banks. Noticed the low users' vigilance and high similarity between \tool and normal power bank, the attacker can return his/her modified power bank (as \tool) to a power bank station, mixing \tool into the circulation of shared power banks. The next user who unconsciously connect his/her device to \tool can find his/her devices being charged as normal.
Like many outdoor mobile phone usage scenarios, the victim may complete payment with showing payment code. In this process, the attacker can preemptively extract and use the victim's payment code to make another payment unknown to the victim, which threatens victim's property.


In order to check what can be obtained by \tool, we conduct a user study in order to check the performance of \tool in Privacy extraction mode. We invited 6 volunteers to our lab room. Volunteers take turns to operate their phones for half an hour while \tool is connected. Before the study, we did not reveal the function of \tool and asked the volunteers to connect their phones to \tool as a normal power bank (ignoring its larger size). 
After the study, we told them our attack, checked recorded videos together, and confirmed the consent to study the information in recorded videos, Subject to efficiency reasons, we ask them to use phones like when they use the shared power bank outside.


\subsubsection{Result}

After collecting the videos, we used both automatic and manual methods to analyze the video. In the automatic analysis, we use a script to simply perform OCR recognition to each frame in the recorded video and store all OCR results into a database in a format of ``frame number - text by OCR''. At last, there are 38329 pieces of data collected among 6 volunteers. In the manual analysis, we replay the recorded video and extract sensitive information manually.

In terms of automatic analysis, OCR results of all frames are already stored in the database and can be used at hand. With these results, we can learn what issues the victim is browsing. Additionally, some keywords often appear with users' input data because they are often used as labels of input boxes, such as ``account'', ``username''. Such keywords are more likely to lead us to discover user-specific data. For example, in Table~\ref{tab:ocr_keyword_example}, searching with ``account'' as the keyword, victim's accounts can be detected as shown in Fig. The frame number is the position of this frame in the recorded video, which indicates a target for manual analysis for further data extraction.

\begin{table*}[hbtp]
	\centering
	\begin{tabular}{|l|l|l|l|}
		\hline
		Keyword  & Text                                                                                                                          & Name                           & Frame Number \\ \hline
		username & X 8B cas.******.edu.cn Username: 117***18 Password:                                                                           & \textless{}user1\textgreater{} & 385          \\ \hline
		username & Login Weibo Login with SMS and verification code ...... +86 151****4587 Get verification code Login with username \& password & \textless{}user5\textgreater{} & 1947         \\ \hline
		username & QQ 14*****50| Login with phone number New user registration 2345678 9 0                                                       & \textless{}user3\textgreater{} & 4308         \\ \hline
		username & connect to *** username h*****l Save account information Open VPN.....                                                        & \textless{}user6\textgreater{} & 7925         \\ \hline
		+86      & Login with phone number ...... +86 186****2483 |                                                                              & \textless{}user1\textgreater{} & 313          \\ \hline
	\end{tabular}
	\caption{Example of searching OCR results with some keywords}
	\label{tab:ocr_keyword_example}
\end{table*}


In terms of manual analysis, data collected are listed in table ~\ref{table:information_extracted}. Accounts of internet applications such as Apple ID, iCloud, Facebook, Twitter, WeChat, QQ, Alipay can be obtained. Additionally, all typing inputs are on the virtual keyboard (including the system keyboard and the built-in security keyboard of the financial app), which can be clearly recorded. Noticed that, we can obtain the plain text of password typed such as the WiFi password and the Alipay password. Furthermore, the received SMS verification code (usually is used to confirm real-name authentication) can be obtained when it appears in the top notification bar. 


In summary, we can no longer directly peek at the user’s password on the lock screen. But if the user unconsciously unlocks the screen, we can still check all information presented on the screen, extract private information including but not limited to social accounts, bank accounts, personal financial situation, etc. It is worth noting that the so-called secure keyboards built into some financial apps just simply disrupt the keyboard sequence and cannot prevent attacks similar to \tool.


\begin{table*}[hbtp]
	\centering
	\begin{tabular}{|c|c|c|c|c|}
		\hline
		Application Column  & Application & Private information leaked                       \\
		\hline
		Finance App         & Alipay      & Alipay account, personal assets(blance)          \\
		\hline
		Social  Finance App & WeChat      & WeChat account, blance, chat history             \\
		\hline
		Social App          & QQ          & QQ account, interpersonal nexus, chat history    \\
		\hline
		Social App          & Twitter     & Twitter account, interpersonal nexus             \\
		\hline
		Social App          & Gmail       & Gmail account, mail records                      \\
		\hline
		Finance App         & ICBC        & ICBC account, password, personal assets(blance)  \\
		\hline
		Finance App         & Paypal      & Paypal account, blance, bank accounts            \\
		\hline
		Tool                & Chrome      & Sites visited                                    \\
		\hline
		Tool                & Health      & personal physical metrics      					 \\
		\hline
	\end{tabular}
	\linebreak
	\caption{Information extracted}\shuqing{Compress.}
	\label{table:information_extracted}
\end{table*}