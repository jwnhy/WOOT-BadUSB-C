\section{Experiment}
\label{sec:experiment}
%\shuqing{Many parts in this section are much more likely to lie in implementation, introducing how \tool works. Doesn't look like experiment.}

%\shuqing{Experiment for devices and case study needed.}

To evaluate the effectiveness of \tool in different modes, we conducted three
experiments on \tool using devices with \ac{USB} Type-C capabilities from different
\acp{OEM}, including a mobile phone, a tablet, and a laptop.

\textbf{Setup.}
%\fengwei{I would suggest to add references for devices such as
%Pi, ATMEGA32U4 board, Yamazawa, etc.}
As mentioned in Section~\ref{sec:badusb},
our \tool only requires common components that are easy to access online or
in any electronic store. Here we chose the following parts to build a
prototype. To begin with, we chose the Raspberry Pi 4B~\cite{pi4b} as the embedded Single Board
Computer inside \tool, which is powerful enough to process video data and has
an onboard WiFi chip. As for the \ac{HID} Emulator, we used an Atmel ATMEGA32U4 board~\cite{atmel}
with \ac{USB} protocol support, which is able to emulate multiple \acp{HID}
with our modified firmware. About the \ac{USB} 3.x Hub, we used one from
UGREEN~\cite{ugreen}, which supports HDMI, \ac{USB} 2.0, and many other exported peripherals.
Apart from these essential parts, we also used an auxiliary power bank to
provide power for the Raspberry Pi and the mobile devices used by the victim.
The image of our \tool prototype can be found in Figure~\ref{fig:armory}.

\circled[text=white,fill=myblue]{\scriptsize{A}} is a HUAWEI mobile phone, the victim's device; \circled[text=white,fill=myblue]{\scriptsize{B}} is a compact look of \tool prototype; \circled[text=white,fill=myblue]{\scriptsize{1}} is the \ac{USB} 3.x Hub; \circled[text=white,fill=myblue]{\scriptsize{2}} is a Raspberry Pi 4B as the Single Board Computer; \circled[text=white,fill=myblue]{\scriptsize{3}} is an auxiliary power bank; \circled[text=white,fill=myblue]{\scriptsize{4}} is the Video Capture Card; \circled[text=white,fill=myblue]{\scriptsize{5}} is an Atmel ATMEGA32U4 board as the \ac{HID} Emulator.

%\fengwei{We need to explain the Figure. What is "A"? What is "B", Where are 1,
%2, 3, 4, and 5?}\hongyi{Is the caption sufficient? If explain here, I feel quite redundant}
%\fengwei{I think it is necessary to explain them in the text again.}

\begin{figure}[t]
	\includegraphics[width=\linewidth]{./Figs/armory_all.png}\\
	\begin{tabular}{ll}
	\circled[text=white,fill=myblue]{\scriptsize{A}} Victim's Device    &\circled[text=white,fill=myblue]{\scriptsize{B}}~\tool\\
	\circled[text=white,fill=myblue]{\footnotesize{1}} \ac{USB} 3.x Hub        &\circled[text=white,fill=myblue]{\footnotesize{2}} Raspberry Pi 4B\\
	\circled[text=white,fill=myblue]{\footnotesize{3}} Auxiliary Power Bank &\circled[text=white,fill=myblue]{\footnotesize{4}} Video Capture\\
	\circled[text=white,fill=myblue]{\footnotesize{5}} ATMEGAA32U4 Board
	\end{tabular}


	\caption{\tool Prototype.}
	\label{fig:armory}
\end{figure}

\subsection{Attack Initialization}
\label{subsec:attack_init}
After the \tool is plugged into the victim's device, there is an initialization process to make sure the \tool is able to carry out subsequent attacks.
\begin{itemize}
	\item \textbf{Screen Mirroring}:
		During our experiment, we noticed that some devices did not mirror their main displays to our \tool by default. In this case, \tool would inject a series of keystrokes to mirror the victim's primary display. On both Windows 10 and Ubuntu (Gnome Desktop), \tool can inject ``Win+P'' (the ``Win'' key is the ``Super'' key on Windows and Ubuntu by default) to switch the display modes for the external display. Figure~\ref{fig:ubuntu_switch} illustrates how these keystrokes work on Ubuntu, similar to that on Windows 10. On MacOS, according to the official manual~\cite{appleman}, \tool can inject \mbox{``Command+F1''} keystrokes to mirror the victim's primary display. In EMUI, there is a special mode called ``Desktop Mode'' that allows users to use their smartphones as desktop computers with external displays. If this mode is enabled, since the victim's ``desktop'' can be obtained, \tool can inject mouse movements to switch to mirror mode.
		\begin{figure}[H]
			\includegraphics[width=\linewidth]{./Figs/ubuntu_switch.png}\\
			\caption{Switching Display Mode on Ubuntu.}
			\label{fig:ubuntu_switch}
		\end{figure}
	\item \textbf{Dismiss Notification}:
		In our experiment, we found that some devices would notify the user of the presence of an external display. During our experiment, the following notifications were raised:
		\begin{itemize}
		 \item On Lenovo Xiaoxin Pro 13, it raised a pop-up asking user to choose the functionality of the external display.
		 \item On HUAWEI P30, it showed a status bar indicator and a persistent notification about the external display.
		 \item On iPad Pro (the third generation), if iPad is locked, it also showed a blue status bar indicator and a notification about \ac{USB} accessories.
		\end{itemize}
		To avoid being discovered by the victim, \tool can inject keystrokes and mouse movements to eliminate some of these notifications.
		For example, in the first case, \tool is able to inject a series of mouse movements to set the display to mirror mode, thereby eliminating the pop-up.
\end{itemize}
\begin{comment}
\begin{table*}[]
	\begin{tabular}{|c|c|c|c|}
		\hline
		Device                                                               & Operating System                                                                                         & Notification              & Dismiss Action                                            \\ \hline
		\begin{tabular}[c]{@{}c@{}}Lenovo Xiaoxin Pro 13\\ 2020\end{tabular} & Windows 10 (OS Build 18363.1379)                                                                         & Pop-up                    & ``Win+P''-\textgreater{}Mouse click on ``Duplicate'' Option \\ \hline
		\multirow{2}{*}{HUAWEI P30 (ELE-AL00)}                               & \multirow{2}{*}{\begin{tabular}[c]{@{}c@{}}EMUI 11.0.0.135(C00E128R2P5)\\ Android 10 based\end{tabular}} & Status bar indicator      & N/A                                                       \\ \cline{3-4}
		&                                                                                                          & Persistent notifications  & N/A                                                       \\ \hline
		\multirow{2}{*}{iPad Pro (3rd generation)}                           & \multirow{2}{*}{iOS 14.4 (Build 18D52)}                                                                  & Status bar indicator      & N/A                                                       \\ \cline{3-4}
		&                                                                                                          & Pop-up (only when locked) & N/A                                                       \\ \hline
	\end{tabular}
	\linebreak
	\caption{Notifications about \tool}
	\label{tab:notification}
\end{table*}
\end{comment}

\subsection{HID Emulator Mode}

In the experiment of HID emulator mode, we used {Lenovo Xiaoxin Pro 13
2020}, a PC with Windows 10 \mbox{(OS Build 18363.1379)} and Ubuntu 20.04.2 LTS with two \ac{USB} Type-C interfaces as the
target devices. During this experiment, \tool disguised itself as a normal keyboard and an external display as a feedback channel. When first plugged in, the victim's device raised a pop-up window asking the victim which mode should the new screen be set to. \tool immediately injected a series of mouse movements and clicks to set itself as a mirror to the primary display. Thus we had successfully completed the attack initialization. Then we tested three scripts, ranging from reverse shell backdoor to malware payload execution, all of them resulted in success.
Compared with original BadUSB attacks like Rubber Ducky, our \tool can provide attackers real-time feedback, which allows attackers to conduct more accurate attacks.
%This also warned us that we need more thorough defense against such BadUSB
%attack.


%\fengwei{do we need a reference for OCR?}
%\hongyi{we have discussed this and decide no need.}
\subsection{Video Capture Mode}

During the experiment of privacy extraction via our video capture mode, we chose \mbox{\textit{HUAWEI P30 (ELE-AL00)}}, a
smartphone in \mbox{EMUI 11.0.0.135(C00E128R2P5)} \mbox{(Android 10.0 based)} with a \ac{USB} Type-C interface, as the target device.
In the privacy extraction experiment, \tool passively captured video from the victim's device and used \textit{OpenCV} to identify the sensitive information from the video stream.  When the victim viewed text or photos with text, \tool used the techniques of \ac{OCR} to extract text from corresponding video frames.
In this experiment, the attacker successfully extracted text such as name, address, ID number, and other sensitive personal information. We also tested the payment code extractor, which enables an attacker to identify payment code in the video stream and performs transactions
without the password. During our experiment, we noticed that this HUAWEI device supports ``Desktop Mode'' for its external display, which enable users to use their devices as if they were using a computer with an external display.
If \tool is set into this mode, \tool is able to inject mouse movements to set itself into mirroring the primary display, as we discussed in Section~\ref{subsec:attack_init}.
As this is also a part of our case study, more details about
the extracted sensitive data can be found in Section~\ref{subsec:case_study} and
Table~\ref{table:information_extracted}.

%\fengwei{I don't understand the following sentence. Both modes need to capture
%the video stream. Don't understand why it is more power-efficient.}
%\hongyi{Reasonable, changed to other advantage}
Note that the video capture mode only needs to
process the victim's video stream locally, it does not need to transmit the real-time video back to the attacker, which is useful when the network connection between \tool and the attacker is not stable.
%This mode is useful when there is no stable network connection between \tool and the attacker.

\subsection{Full Control Mode}
%\fengwei{In BadUSB-C section, remote control model is after privacy extraction mode.}

To test the capability of the full control mode, we chose iPad Pro (3rd generation),
a tablet running iOS 14.4 (Build 18D52) with a \ac{USB} Type-C interface,
as the target device in this experiment.
In addition to disguising themselves as normal \acp{HID} like a conventional BadUSB~\cite{badusb}, \tool also transmitted the real-time video stream from the target device to the attacker via WiFi.
After establishing the connection, the attacker performed a series of actions to test the capability of \tool.
In the beginning, the attacker accessed the album application on the iPad and obtained all the photos inside.
After that, the attacker sent messages via the victim's account. At last, the attacker performed a transaction using a financial application.
All of these tests resulted in success.

Through this experiment, we have found that with video transmission and mouse
emulation, \tool extensively expanded the attack capability of BadUSB,
especially in mobile devices. In short, we have achieved the complete hijack of the victim's
device in this experiment.

\subsection{Case study}
\label{subsec:case_study}
\tool can be used in various attack scenarios, ranging from mobile devices to PC devices.
For example, \tool can be attached to the power station, which provides USB 3.x hubs, in the airport to perform attacks.
Most people charge laptops or smartphones in the power station in emergency, with negligence of security.
In the following paragraphs, we will demonstrate the attack scenarios of \tool using sharing power bank as an example.

\subsubsection{Background}

We first introduce the technical background of our case study, sharing power
banks and QR code payment.

\textbf{Sharing Power Banks}. Sharing power banks provide users with short-term
rental of power banks. The company deploys power bank stations in the city and
users can rent a power bank from any of the power bank stations, charge their
device on the trip, return the rented power bank to the near stations, and pay
the rental fee.

Power bank sharing is a popular service in Asia, power bank stations are deployed in markets, stores, and even newsstands.
For example, Figure~\ref{fig:PBS_products} are photos of two power bank stations in China, which are taken outside of a supermarket. Brick~\cite{Brick} is also such a power bank sharing service provider from Sweden.
It provides power bank rental service all over Sweden and is planning on expanding its service to entire Europe.
\begin{figure}[t]
	\centering
	\includegraphics[width=.45 \linewidth, height=.45 \linewidth]{./Figs/PBS_mt.png}
	\includegraphics[width=.45 \linewidth, height=.45 \linewidth]{./Figs/PBS_xd.png}
	\caption{Two Power Bank Stations.}
	\label{fig:PBS_products}
\end{figure}

%\shuqing{May use statistics (instead of concrete examples) to explain it.}

Sharing power banks bring convenience but also security issues.
We noticed that most of the power bank stations do not check the integrity of the power bank during the rental process, and users rarely check the power banks carefully when connecting the power bank to their devices.
An attacker can tamper with a rented power bank and return it to a power bank station, thereby posing a potential threat to subsequent users.


\textbf{QR Code Payment}.
QR code payment is a new payment method popular in Asia.
The most famous cases are WeChat Pay~\cite{Wechat-pay} and Alipay~\cite{AliPay}.
QR code payment provides merchants and customers with a convenient offline payment method while ensuring equivalent security as credit cards.
As illustrated in Figure~\ref{fig:qr_payment_procedure}, QR code payments are typically performed in the following steps:
(1) The customer shows the QR payment code to the merchant on the mobile device.
The QR code is encoded with a globally unique ID to identify the customer's account.
(2) The merchant scans the QR payment code and charges the corresponding amount.
By showing this QR code, the customer authorizes the proceeding transaction.
(3) After confirmation, the payment service provider proceeds with this transaction and returns the payment result to both the merchant and the customer.

\begin{figure}[t]
	\centering
	\includegraphics[width=\linewidth]{./Figs/qr_code_payment.png}
	\caption{Bar/QR Code Payment Procedure.}
	\label{fig:qr_payment_procedure}
\end{figure}


Next, we explain a payment method called micropayment.
A micropayment is pre-determined by the payment service provider according to the threshold in the user agreement.
In the real world, payment service providers use different rules for micropayment purchases.
For example, WeChat Pay~\cite{Wechat-pay} treats transactions under US\textdollar 154 as micropayments.
Different from the typical payment procedure, when marking micropayments, confirmations can be automatically applied without the customer's permission, which aims to provide convenience for both merchant and customer.
If the victim's payment code is leaked to the attacker, the attacker can use the code to authorize multiple micropayments without permission.
In order to prevent this situation, both WeChat Pay~\cite{Wechat-pay} and Alipay~\cite{AliPay} have designed a mechanism to refersh the payment code every minute.
This is sufficient to prevent attacks like opportunistic theft of a payment code, but cannot prevent real-time attacks like our \tool.
In summary, the QR payment code is highly sensitive on users' devices. Th following case study is about how to use an attacker-crafted power bank via \tool to obtain this code.

\subsubsection{Attack Scenario}
\label{subsec:attack-scenario}

In this part, we will introduce a real-life attack scenario to show that our \tool is a practical offensive tool.
This scenario can be broken down into the following steps.

\begin{enumerate}[I. ]
	\item The attacker rents a power bank from one of the power bank stations and replaces the internal components with \tool.
	\item After the modification, an attacker-crafted power bank is returned to the rental station in crowded areas like airports or railway stations, which increases the probability of success.
	\item A user borrows the modified power bank and connects it to his/her own device, becoming a victim of \tool.
	\item The attacker can now fully control the victim's device and can use different modes for various attacks.
\end{enumerate}

Next, we summarize the possible threats toward the victim under different modes.
First, under \ac{HID} emulator mode, the attacker is able to implant malware and backdoor scripts into the victim's device. Moreover, using video capture mode, once the victim accesses his/her sensitive data such as QR payment code or photos, the sensitive data will be immediately transmitted to the attacker via \tool. Lastly, with full control mode, the attacker can completely control over the victim's device and can perform any operations on the victim's device.

\subsubsection{Experiment}

We conducted experiments to further validate the capabilities of \tool
in the attack scenario introduced in Section~\ref{subsec:attack-scenario}.
11 applications were selected and tested on \mbox{\textit{HUAWEI P30 (ELE-AL00)}} step by step:
(1) Login in with a test account.
(2) Keep the default settings.
(3) Attach attacker's \tool to the test device.
(4) Simulate victim's daily usage of the application.

During this experiment, we tested full control mode which actively obtained sensitive information and video capture mode which passively obtained the sensitive information visited by the victim. We noticed that, most sensitive information could be obtained directly through full control mode without victim's interaction, but there was also certain information that had to be obtained passively through video capture mode. For example, as illustrated in Table~\ref{table:information_extracted}, in most applications, information like ``Account'' can be obtained directly while information like ``Payment Password'' cannot be obtained until the victim inputs his/her password.

\subsubsection{Result}
In summary, \tool is able to actively obtain most sensitive information from the screen, such as browsing history, personal account, phone number without victim's interaction. There also exists information such as payment password that is not available immediately, but such a piece of information can be obtained once the victim inputs it. The obtained information can be used to guess the victim's lock screen password and poses further threat to victim's privacy.

\begin{table*}[t]
	\centering
	\begin{tabular}{|c|c|l|l|c|c|}
		\hline
		\multirow{2}{*}{\textbf{Category} } & \multirow{2}{*}{ \textbf{Application} } & \multicolumn{2}{c|}{\textbf{Leaked Sensitive Information}} \\
															\cline{3-4}
											&				& \textbf{Without Victim's Interaction}						& \textbf{With Victim's Interaction} \\
		\hline
		Social \& Finance App 				& WeChat (8.0.1)      & Account, Financial Status, Chat History, Payment Code   		& Payment Password \\
		\hline
		\multirow{2}{*}{Social App}
											%& QQ          & Account, Contacts, Chat History   	& \\
											%\cline{2-4}
							       			& WhatsApp (2.21.5.18)    & Account, Contacts, Chat History, Phone Number    & <None> \\
											\cline{2-4}
							       			& Facebook (309.0.0.47.119)   & Account, Posts, Contacts           & <None> \\
		\hline
		\multirow{3}{*}{Finance App}       	& Alipay (10.2.15.9500)      & Account, Financial Status, Payment Code         				& Payment Password \\
											\cline{2-4}
											& Cash App (3.35.1)   & Email, Phone Number, Cash Balance							& Payment Password \\
											\cline{2-4}
											& Paypal (7.38.1)      & Account, PayPal Balance     								& Payment Password \\
		\hline
		\multirow{1}{*}{Shopping App}		& Amazon Shopping (22.6.0.100)  & Account, Orders, Shopping Cart         				& <None> \\
											%\cline{2-4}
		                					%& Taobao      & Personal physical metrics      						& <None> \\
		\hline
		\multirow{2}{*}{Tool}               & Chrome (89.0.4389.72)      & Browsing History                                	& <None> \\
											\cline{2-4}
		                					& Health (11.0.5.508)      & Personal Health Metrics      						& <None> \\
		\hline
		\multirow{3}{*}{System}             &  Messages (11.0.1.430)   & Contacts, Chat History				 &  <None> \\
											\cline{2-4}
											& Settings (11.0.0.300) - WiFi   & WiFi SSID                                	&  WiFi Password \\
											\cline{2-4}
		                					& Settings (11.0.0.300) - VPN    & VPN Address, VPN Account, VPN Password      						& <None> \\
		\hline
	\end{tabular}
	\linebreak
	\caption{Sensitive Information Leaked From Applications.}
	\label{table:information_extracted}
\end{table*}


%%%%%%%%%%%%%%%%%%%%%%%%%%%%%%%%%%%%%%%%%%%%%%%%%%%%%%%%%%%%%%%%%%%%%%%%%%%%%%%%%%%%%%%%
% It is unfortunate that the user study is not part of the paper anymore.
