\section{Experiment}
\label{sec:experiment}
%\shuqing{Many parts in this section are much more likely to lie in implementation, introducing how \tool works. Doesn't look like experiment.}

%\shuqing{Experiment for devices and case study needed.}

To evaluate the effectiveness of \tool in different modes, we conducted three
experiments on \tool using devices with USB Type-C capabilities from different
OEMs, including a mobile phone, a tablet, and a laptop.

\textit{Setup.} 
%\fengwei{I would suggest to add references for devices such as
%Pi, ATMEGA32U4 board, Yamazawa, etc.} 
As mentioned in Section~\ref{sec:badusb},
our \tool only requires common components that are easily accessible online or
in any electronic store. Here we choose the following parts to build a
prototype. To begin with, we choose the Raspberry Pi 4B~\cite{pi4b} as the embedded Single Board 
Computer inside \tool, which is powerful enough to process video data and has
an onboard WiFi chip. As for the HID Emulator, we use an Atmel ATMEGA32U4 board~\cite{atmel}
with USB protocol support, which is able to emulate multiple HID devices
with our modified firmware. About the USB 3.x Hub, we use one from the
UGREEN~\cite{ugreen}, which supports HDMI, USB 2.0, and many other exported peripherals.
Apart from these essential parts, we also use an auxiliary power-bank to
provide power for the Raspberry Pi and the mobile devices used by the victim.
The image of our prototype \tool can be found in Figure~\ref{fig:armory}.
\mycircled{A} is a Huawei mobile phone as the victim's device; \mycircled{B} is a compact look of \tool prototype; \mycircled{1} is the USB 3.x Hub; \mycircled{2} is a Raspberry Pi 4B as the Single Board Computer; \mycircled{3} is an auxiliary power bank; \mycircled{4} is the Video Capture Card; \mycircled{5} is Atmel ATMEGA32U4 board as the HID Emulator. 
%\fengwei{We need to explain the Figure. What is "A"? What is "B", Where are 1,
%2, 3, 4, and 5?}\hongyi{Is the caption sufficient? If explain here, I feel quite redundant}
\fengwei{I think it is necessary to explain them in the text again.}


\begin{figure}[t]
	\includegraphics[width=.98\linewidth]{./Figs/armory_all.png}\\
	\begin{tabular}{ll}
	\mycircled{A}Victim's Device    &\mycircled{B}\tool\\
	\mycircled{1}USB 3.x Hub        &\mycircled{2}Raspberry Pi 4B\\
	\mycircled{3}Auxiliary Power Bank &\mycircled{4}Video Capture\\
	\mycircled{5}ATMEGAA32U4 Board
	\end{tabular}


	\caption{\tool}
	\label{fig:armory}
\end{figure}

\subsection{Scripting mode}

In the experiment of scripting mode, we used \textit{Lenovo Xiaoxin Pro 13
2020}, a PC in Windows 10/Ubuntu OS with two USB Type-C interfaces as the
target device.  During this experiment, \tool disguised itself as a normal
keyboard and a home-brewed version of GoodUSB~\cite{tian2015defending} is
deployed to test the defense bypassing. We have tried to deploy the original
GoodUSB on the target device; as the GoodUSB is proposed in 2015, the code is
outdated and cannot be executed at modern OS; so we have to home-brew a similar
defense mechanism based on its paper~\cite{tian2015defending}.  In the
beginning, our home-brewed version of GoodUSB asked the victim to manually complete
a CAPTCHA to authorize the new USB device, a.k.a our \tool. But at this time,
by the design of GoodUSB, \fengwei{The following sentence is hard to
understand.}\hongyi{Fixed, removed redundant description} our \tool can already perform limited keystroke to complete the
CAPTCHA with the video stream from \tool. Up
to this point, the bypass of defense like GoodUSB is completed as our \tool
gains full function from the GoodUSB and becomes a trusted device. Then, our
\tool works similar to the original BadUSB, inserting keystroke to perform
arbitrary execution on the laptop. We tested three scripts, ranging from
reverse shell backdoor to malware payload execution, all of which resulting in
success.  With this experiment, we have proven the defense bypassing is successful
using \tool.
%This also warned us that we need more thorough defense against such BadUSB
%attack.



\subsection{Privacy extraction mode}

During the experiment of privacy extraction, we chose \textit{HUAWEI P30}, a
smartphone in EMUI 9.1 (Android 9.0 based) with a USB Type-C interface, as the
target device.  In the privacy extraction mode, \tool passively captured video
from the victim's device and used \textit{OpenCV} to identify the valuable
information from video stream.  When the victim viewed text or photos with
text, \tool used the techniques of Optical Character Recognition (OCR) to
extract text from corresponding video frames.  In this experiment, the attacker had
successfully extracted text like name, address, ID number, and other sensitive
personal information. We also tested the payment code extractor, which the enables
attacker to identify payment code in the video stream and perform transactions
without the password. As this is also a part of our case study, more details about
the data extracted can be found in Section~\ref{subsec:case_study} and
Table~\ref{table:information_extracted}.

\fengwei{I don't understand the following sentence. Both modes need to capture
the video stream. Don't understand why it is more power-efficient.}
\hongyi{Reasonable, changed to other advantage}
After the
experiment, we have found that as privacy extraction mode only needs to
process the victim's video stream locally, it does not need to transmit the real-time video back to the attacker. This mode is useful when there is no stable network connection between \tool and the attacker.

\subsection{Remote control mode}
\fengwei{In BadUSB-C section, remote control model is after privacy extraction mode.}

To test the capability of remote control mode, we chose \textit{iPad Pro (3rd
	generation)}, a tablet in iOS 14.3 with a USB Type-C interface, as the target
device in this experiment.  Besides disguising as normal HID devices like a
conventional BadUSB~\cite{badusb}, \tool also transmitted real-time video
stream from the target device to the attacker via WiFi.  After the connection
establishment, the attacker performed a series actions to test the capability of
\tool. In the beginning, the attacker accessed album application on the iPad and
obtained all the photos inside. After that, the attacker sent messages via victim's
account. At last, the attacker performed a transaction using the
financial application. All of these tests resulted in success.

Through this experiment, we have found that with video transmission and mouse
emulation, \tool extensively expanded the attack capability of BadUSB,
especially in mobile devices. We have archived the complete hijack of the victim's
device in this experiment.

\subsection{Case study}
\label{subsec:case_study}
\subsubsection{Background}

We first introduce the technical background of our case study, sharing power
bank service and QR code payment.

\textbf{Sharing Power Bank}.  Sharing power bank provides users with short-term
rental of power banks.  The company deploys power bank stations in the city and
users can rent a power bank from any of the power bank station, charge their
device on the trip, return the rented power bank to the near station, and pay
the rental fee.

For example, Brick~\cite{Brick} is such a power bank sharing service
provider from Sweden. It provides power bank rental service all over Sweden
and is planning on expanding its service around Europe. Moreover, this
service is even more popular in Asia, power bank stations are deployed in
markets, stores, and even newsstands.
%This suggests that this service is becoming more and more common around the global, especially in Asia and Europe.
\shuqing{I think we should paraphrase instead of using the descriptions on the website directly. Just to explain \textbf{what we need}.}

\fengwei{Explain the figure in the text.}
\begin{figure}[t]
	\centering
	\includegraphics[width=.45 \linewidth, height=.45 \linewidth]{./Figs/PBS_mt.png}
	\includegraphics[width=.45 \linewidth, height=.45 \linewidth]{./Figs/PBS_xd.png}
	\caption{Two Power Bank Stations}
	\label{fig:PBS_products}
\end{figure}

\yechang{add hyperlink or reference to Brick's website or Brick App on App Store/Google Play.}

\yechang{Add more examples of power bank sharing to show that it is widely used?}
\shuqing{May use statistics (instead of concrete examples) to explain it.}

Not only does sharing power bank provides convenience to users, but it also
brings security issues.  We noticed that most of the power bank stations do not
check the integrity of power banks during the rental process, and users are
hardly cautious to check the power banks when connecting their devices.  An
attacker is able to tamper rented power banks and return them to a power bank
station causing a potential threat to subsequent users.


\textbf{QR Code Payment}.
QR code payment is a new type of payment method that is popular in Asia. Its most well-known cases are WeChat Pay~\cite{Wechat-pay} and Alipay~\cite{AliPay}. QR code payment provides merchant and client a convenient way of offline payment while ensuring equivalent security as the credit card.
As illustrated in Figure~\ref{fig:qr_payment_procedure}, QR code payment is typically performed in the following steps:
\ding{182} The client presents the payment QR code on the mobile device to the merchant.
The QR code is encoded with a globally unique ID to identify the client's account.
\ding{183} The merchant scans the payment QR code and charges the corresponding amount of money.
By presenting this QR code, the client authorizes the proceeding transaction.
\ding{184} After confirmation, the payment service provider proceed with this transaction and return the payment result to both the merchant and the client.
\shuqing{I think this paragraph can be shorter. No need to explain so many details.}
\hongyi{Dont know how to be more concise.}

\begin{figure}[t]
	\centering
	\includegraphics[width=\linewidth]{./Figs/qr_code_payment.png}
	\caption{Bar/QR Code Payment Procedure}
	\label{fig:qr_payment_procedure}
\end{figure}


In the real-world scenario, some payment service providers use special rules
for micropayment purchases.  A micropayment is pre-determined by the payment
service provider with thresholds in the user agreement.  For example, WeChat
Pay~\cite{Wechat-pay} regards transaction under USD \textdollar 154 as
micropayments.  Different from a typical payment procedure, when micropayments
are made, confirmations can be applied automatically without clients'
permission which aims to provide convenience to both merchant and client.  If a
victim's payment code is leaked to the attackers, they can use that code to
authorize multiple micropayments without permission.  In order to prevent such
cases, both WeChat Pay~\cite{Wechat-pay} and Alipay~\cite{AliPay} has designed a refreshing mechanism for the
payment code, which is to refresh the QR code every minute. This is sufficient
to stop an attack like sneak shots but unable to stop a real-time attack like our
\tool.  In summary, the payment QR code is highly sensitive on users' devices
and our following case study is about how to obtain this code using an
attacker-crafted power bank via \tool.

\subsubsection{Attack Scenario}

In this part, we will introduce a real-life attack scenario to show that our
\tool is a practical offensive tool.  This scenario can be decomposed into the
following steps.

\begin{enumerate}[I. ]
	\item The attacker rents a power bank from one of the power bank stations and replaces the internal components with \tool.
	\item After the modification, an attacker-crafted power bank is returned to a rental station in a crowded area like an airport or railway station, which increases the probability of success.
	\item a user borrows the modified power bank and connects it to her own device, becoming the victim of \tool.
	\item The attacker now has complete control over the victim and can perform various attacks using different modes.
\end{enumerate}

Here we summarize the possible threats toward user under different modes. \hongyi{Better way to express this?}
First, under scripting mode, the attacker is able to implant malware and backdoor script into victims' devices. Moreover, using privacy extraction mode, once the victims access their private data like QR payment  code or album, their private data will be immediately transmitted via \tool to the attacker. Lastly, with remote control mode, the attacker has complete control over the victims' device and can do anything they want.

\shuqing{Background is longer than real user study.}
As the functionality and effectiveness of the scripting mode and the remote control mode are rather clear, the former works similar to the original BadUSB, the latter hijacks the victim's device completely.\hongyi{I dont know if it's a good enough reason}
Here we only validate the usability of privacy extraction mode and conducted a user study.
We invited 10 volunteers to participate, who are unconscious of \tool details.
Before the experiment, we disclose to the volunteers how their data might be used and request permission from both the volunteers and the institutional ethics review boards.
During the experiment, volunteers took turns to use their phones for half an hour with \tool connected, which are considered as a normal power bank.
To obtain data close to real life, they are requested to use phones just like when they use the shared power bank outside.
After the experiment, we introduced to them our attack, analyze the screen recordings together to make sure their personal data is not at risk.
\shuqing{Do we need to mention that agreement the conference required here?}
\hongyi{I'm also quite concern about this.}


\subsubsection{Result}
After collecting the videos, we analyzed the video both automatically and manually.
During the automatic analysis, we used scripts to perform OCR recognition for each frame in the recorded video and stored all of the OCR results in a database.
At last, there are 94058 pieces of data collected among 10 volunteers.
\shuqing{The data may need to be updated.}
With these results, we could learn what content the victim was browsing.
Additionally, some keywords such as \textit{account, username, and password} often appear with users' input data because they are often used as labels of input boxes.
Such keywords are more likely to lead us to user-specific data.
For example, when we searched with \textit{account} as a keyword, victim's accounts can be found in the database, as shown in the Table~\ref{tab:ocr_keyword_example}.
\shuqing{Statistics.}
The frame number is the position of this frame in the recorded video, which indicates a target for manual analysis for further data extraction.

\begin{table*}[t]
	\centering
	\begin{tabular}{|l|l|l|l|}
		\hline
		Keyword  & Text                                                                                                                          & Name                           & Frame Number \\ \hline
		username & X 8B cas.******.edu.cn Username: 11****18 Password:                                                                           & \textless{}user1\textgreater{} & 385          \\ \hline
		username & Login Weibo Login with SMS and verification code ...... +86 151****4587 & \textless{}user5\textgreater{} & 1947         \\ \hline
		username & QQ 14*****50| Login with phone number New user registration 2345678 9 0                                                       & \textless{}user3\textgreater{} & 4308         \\ \hline
		username & connect to *** username h*****l Save account information Open VPN.....                                                        & \textless{}user6\textgreater{} & 7925         \\ \hline
		+86      & Login with phone number ...... +86 186****2483 |                                                                              & \textless{}user1\textgreater{} & 313          \\ \hline
	\end{tabular}
	\linebreak
	\caption{Example of searching OCR results with some keywords}
	\label{tab:ocr_keyword_example}
\end{table*}


In the manual analysis, we replayed the recorded video and extracted sensitive information.
The data we collected are listed in Table~\ref{table:information_extracted}.
Accounts of internet applications such as Apple, iCloud, Facebook, Twitter, etc. can be obtained.
Moreover, all of the typing inputs on the virtual keyboard, including the system keyboard and the built-in security keyboard of the financial applications, can be clearly recorded.
We can obtain the plain text of passwords such as WiFi passwords.
Furthermore, the received SMS verification code (usually used to confirm real-name authentication) can be obtained when it appears in the top notification bar.

In summary, though we can't directly obtain the user's password on the lock screen, we can still check all of the information presented on the screen, extract private information including but not limited to social accounts, bank accounts, personal financial situation, etc., if the user unconsciously unlocks the screen.
It is worth mentioning that, those \textit{secure keyboards} built in some financial apps just disrupt keyboard sequences, they can't prevent attacks similar as \tool.

\begin{table*}[t]
	\centering
	\begin{tabular}{|c|c|c|c|c|}
		\hline
		Application Column  & Application & Private information leaked                       \\
		\hline
		Finance App         & Alipay      & Alipay account, personal assets(blance)          \\
		\hline
		Social  Finance App & WeChat      & WeChat account, blance, chat history             \\
		\hline
		Social App          & QQ          & QQ account, interpersonal nexus, chat history    \\
		\hline
		Social App          & Twitter     & Twitter account, interpersonal nexus             \\
		\hline
		Social App          & Gmail       & Gmail account, mail records                      \\
		\hline
		Finance App         & ICBC        & ICBC account, password, personal assets(blance)  \\
		\hline
		Finance App         & Paypal      & Paypal account, blance, bank accounts            \\
		\hline
		Tool                & Chrome      & Sites visited                                    \\
		\hline
		Tool                & Health      & personal physical metrics      					 \\
		\hline
	\end{tabular}
	\linebreak
	\caption{Information extracted}\shuqing{Compress.}
	\label{table:information_extracted}
\end{table*}
