\section{Experiment}
\label{section:experiment}

Overview, the experiment equipment consists of
a USB Type-C hub with HDMI and USB port,
a raspberry pi 4B with a Wi-Fi chip onboard,
a video capture card adapting HDMI signal to USB camera,
and a normal powerbank with dual outputs.

The normal powerbank is used to supply a piece of raspberry pi 4B and
charge victim's device.
The USB-C hub adapts USB Type-C to HDMI port and USB port.
 The whole equipment exposed only one type-c cable (from USB-C hub),
and the equipment will charge his or her device through this cable,
which has no difference from ordinary powerbanks.


\subsection{Real-time monitoring}

When connected to network,
the raspberry pi can use ``ffmpeg'' to encode video stream from victim device,
and then use ``netcat'' to send encoded video stream to a intermediate transit server.
By this means, the attacker is able to monitor the victim devices in real-time.

\begin{lstlisting}[caption={Precondition Checking for the Selfdestruct Call},label=lst1:mxm]
    nc -nlkvp 1234 | nc -nlkvp 4321

\end{lstlisting}


\subsection{Private information extraction}

Real-time monitoring is ideal for the attacker to grab all information shown on victim devices,
 but it requires high power consumption to encode video streams and stable network connections.
In some situations, high power consumption will increase the posibilities of alerting the victim,
and it is hard to guarantee stable network connections.

Considering that, extracting only high-value information for attacker (or highly sensitive information for victim) is more efficient.
The raspberry pi can use ``opencv'' to identify information from video stream.
For example, extracting the victim's name, photo, address, and ID number when victim is viewing his or her photos including a photo of his or her ID card,
extracting the victim's payment information or payment code when victim is showing his or her payment information.
Here is an example of extracting payment code with opencv library in listing~\ref{lst2:qr},

\begin{lstlisting}[caption={python script for extracting payment code of victim},label=lst2:qr]
    import cv2
    import pyzbar.pyzbar as pyzbar
    import qrcode
    import requests
    def decodeDisplay(video):
        global current
        gray = cv2.cvtColor(video, cv2.COLOR_BGR2GRAY)
        barcodes = pyzbar.decode(gray)
        for barcode in barcodes:
            barcodeData = barcode.data.decode("utf-8")
            barcodeType = barcode.type
            img = qrcode.make(barcodeData)
            requests.post(
                'http://attacker',
                data={'code': barcodeData}
            )
    if __name__ == '__main__':
        # 0 is the camera ID
        cam = cv2.VideoCapture(0)
        while True:
            ret, frame = cam.read()
            decodeDisplay(frame)
            if cv2.waitKey(5) == 27:
                break
        cam.release()
        cv2.destroyAllWindows()
\end{lstlisting}
