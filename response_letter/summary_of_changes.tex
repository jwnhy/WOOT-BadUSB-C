\documentclass[journal,12pt,onecolumn,draftclsnofoot,]{IEEEtran}
\newcommand{\TheName}{\textsc{Ninja}}

\usepackage{pifont} 
\usepackage{url}
\def\UrlBreaks{\do\/\do-}

\begin{document}
\title{Summary of Changes}
\maketitle


We would like to express our gratefulness to the reviewers for your 
precious time on reviewing our submission. Here, we present the
motivation of submitting a journal paper based on our USENIX
Security 2017 conference paper and summarize the major updates
and improvements. Then, we provide a list of the detailed changes
in our IEEE TIFS submission.

\section{Changes Between IEEE TIFS Submission and USENIX Security 2017 Paper}

In our USENIX Security 2017 conference paper~\cite{ninja:usenix17}, we proposed
\TheName{}, which is a transparent malware analysis system on ARM platform.
It reduces the footprints and artifacts of an analysis system and provides a
more transparent execution environment for the malware analyst. 

However, in the scenario of continuous malware analysis, it is also important
to achieve a fast system restoration. Thus, we design a fast
restoration mechanism which improves the existing restoration
systems~\cite{barebox:acsac11,barecloud:usenix14,malt:sp15,bolt:acsac17}. 
Specifically, we implement selective memory restoration, runtime file system
restoration, and complete context restoration. The selective memory restoration
leverages hardware-based tracing to record the changed memory addresses and
helps to improve the efficiency of memory restoration. The runtime file system
restoration uses existing runtime file system switching technology to efficiently switch
between alternative file systems, while the complete context restoration performs
a complete system register restoration after each analysis session.

Moreover, we extend the trace subsystem of \TheName{} to support data address tracing.
The data address tracing records the related data address of each load/restore
instruction and helps an analyst to learn additional information
about the current running status of a target malware. As an example, 
the analyst may easily infer the plaintext or encryption keys in AES algorithm
with the data addresses~\cite{armageddon:usenix16}. Moreover, the data address 
information is also helpful for dynamic taint analysis~\cite{sokda:sp10} and 
the selective memory restoration discussed above.
In light of this, we add data address tracing to the trace subsystem and implement
it in an NXP i.MX53 Quick Start Board. This implementation
leverages Linux as the rich OS instead of Android (which we implemented in our USENIX Security conference version) and shows the OS-agnostic feature of \TheName{}.

We also
improve the instruction tracing with address range and process ID filters.
These filters help the analyst focus on the interested instructions and processes.
Additionally, we implement more debugging functions related to the CPU 
speculative execution. The PMU-based functionality may provide more granularities 
for stepping. Moreover, some of these granularities are related to the 
speculative execution feature of the CPU which is abused by recent 
Meltdown~\cite{meltdown:lipp} and Spectre~\cite{spectre:kocher} attacks.
With these additional granularities, the analyst can calculate the Meltdown-
and Spectre- related events like cache miss ratio, number of speculatively
executed instructions, and branch mispredict ratio. These statistic data
are proposed to infer Meltdown and Spectre 
attacks~\cite{detection:israel, detection:capsule8}.

\iffalse
Our case study with the Meltdown attack on the ARM 
platform~\cite{meltdownpoc:cosmin} shows that \TheName{} is capable of detecting 
this new attack. 
\fi

Some contents are also removed due to conciseness and the page limit. 


\section{Detailed Changes}

\subsection{Abstract}
1) Added a discussion of the newly implemented system restoration mechanism.

\subsection{Introduction}
1) Added a brief introduction of the improved system restoration.

2) Added the performance evaluation result of our system restoration mechanism.

3) Added a detailed description of the new contributions of this journal extension 
as compared with our conference paper.

4) Removed the evaluation result of instruction skid.

\subsection{Background}
1) Removed the background about ARM architecture.

2) Removed Figure 1: The ARMv8 and ARMv7 architectures.

% FIXME: What is Futuremark?
3) Removed the survey of Futuremark about the ARM processors in popular smartphones and tablets.

\subsection{Related Work}
1) Added Section 3.3 to discuss the related works on the system restoration.

3) Revised the related works on transparent malware analysis on x86 in Section 3.1.

2) Removed Section 3.3: TrustZone-related systems.

\subsection{System Architecture}
1) Added data address tracing in Section 4.2 (i.e., using another feature
of the ETM to achieve the data address tracing).

2) Added CPU speculative execution related example (i.e., the 
\texttt{INST\_SPEC} event that fires after an instruction is 
speculatively executed) in Section 4.3.

\subsection{Design and Implementation}
1) Added a new implementation testbed, an NXP i.MX53 Quick Start Board.

2) Added instruction address filter in Section 5.3.1 (i.e., using the
address range comparators in the ETM to restrict the trace to
specific memory ranges).

3) Added process ID filter in Section 5.3.1 (i.e., using the context
ID filters in the ETM to restrict the trace to a specific process).

4) Added Section 5.3.4 to describe the detailed data address tracing.

5) Added $6$ more granularities for
stepping debug including branch instruction,
mispredicted branch instruction, L1 data cache read operation,
L1 data cache refill operation, speculatively executing a
load instruction, and speculatively executing a branch instruction 
in Section 5.4.1.

6) Added Table 1: Representative stepping modes in \TheName{}.

7) Added Section 5.5 to describe the improved system restoration including selective memory restoration, runtime file system restoration, and complete context restoration.

8) Revised Section 5.6 for conciseness.

\subsection{Evaluation}
1) Added Section 7.3.2 to evaluate the performance of the newly implemented
system restoration.

2) Added Table 5: Time consumption of system restoration.

3) Revised the description in Section 7.2.1 and Section 7.2.2 for conciseness.

4) Removed Section 7.1: Output of tracing subsystem and Section 7.5: Skid evaluation.

5) Removed Figure 7: Accessing system instruction interface and Figure 8: Memory
mapped interface.

\subsection{Conclusion}
1) Added a description of the improved system restoration mechanism.

\subsection{Appendix}
1) Removed the appendix section.

\bibliographystyle{IEEEtran}
\bibliography{IEEEabrv,./bibliography/zhenyu}


\end{document}

